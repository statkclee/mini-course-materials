
\documentclass[11pt]{article} % 'article' is the document type while '11pt' is an optional argument. there are many document types, and there are many potential options to include for each document type.

\usepackage{geometry}
\geometry{a4paper}     % ... or letterpaper or a4paper or a5paper or ... 
\usepackage{graphicx}
\usepackage{amssymb}
\usepackage{epstopdf}
\usepackage{amsmath}  % this permits text in eqnarray among other benefits
\usepackage{color}          % gives color options
\usepackage{fullpage} % this package makes use of more of each page but is not always installed
\usepackage{natbib}
\usepackage[hangul]{kotex} % Hangul options

\DeclareGraphicsRule{.tif}{png}{.png}{`convert #1 `dirname #1`/`basename #1 .tif`.png}

\title{LaTeX 소개}
\author{David Diez, OpenIntro.org\\
              번역: 이광춘, xwMOOC}
\date{}  % removing the comment makes it produce no date

\begin{document} % this is the formal beginning of the document. the stuff above essentially preloads files and gets prior info.
\maketitle % don't want a title, author, or date listed? comment out this line.
\tableofcontents % compile twice to have the TOC up to date; comment this line out for no TOC
%\setlength{\parindent}{0in} % activate this to have no paragraph indentation
\pagebreak

텍스트를 제어하는 방법에 대해 본 \texttt{latex-intro-kr.tex} 및 \texttt{latex-intro-kr.pdf}를 읽어 친숙해지고, 함께 딸려오는 발표 슬라이드를 미리 읽어보기 강력 추천한다.

\section{일반}

\subsection{표제 절(Header section)}

\texttt{$\backslash$title}, \texttt{$\backslash$author}, \texttt{$\backslash$date} 을 예외로 하고, LaTeX 문서에 있어 이 부분이 종종 무시된다. 항상, 제목, 저자, 가능하면 날짜를 갱신하라. 그렇지 않는 경우 \texttt{$\backslash$maketitle} 명령어를 주석 처리하라. 추가적인 자세한 사항에 대해서는 발표 슬라이드를  참조한다.

\subsection{개요 (Outline)}

명령어 section 과 subsection 으로 장과 절을 만든다. 예를 들어, ``일반'' 은 장이 되고, ``헤더 절 (header section)'' 과 ``개요'' 는 절이 된다.

\subsection{문단 (Paragraphs)}

텍스트 사이에 간단하게 공백 두 줄을 넣으면 새로운 문단이 생성된다. 예를 들어, 이 문단은 ``엔터(enter)'' 키를 두번 치면 종료된다 (.tex 문서 참조) ... 
이것은 새 문단은 아니다 (``엔터(enter)''를 한번만 침).

하지만, 이것은 새 문단이다.  만약 추가 여백이 문단 사이에 필요하면, 역슬래쉬 두개 명령어를 사용하고 나서 ``엔터(enter)'' 키를 두번 친다... \\

PDF 가 알아서 마지막 문단과 이 문단 사이에 공백을 삽입한다. \\


\noindent 특정 문단에 들여쓰기를 하지 않으려면, 이 문단처럼, \texttt{$\backslash$noindent} 명령어를 사용한다.\\

일반적으로, 추가로 많은 공백을                 넣어도 PDF 출력 텍스트에는                            영향이 없다. \texttt{$\backslash$pagebreak} 명령어로를 사용해서 새 페이지를 생성한다.

\pagebreak

\subsection{공백 (Spacing)}
\label{spacing}

수평 텍스트가 \texttt{$\backslash$hspace\{0.3cm\}} \hspace{0.3cm} 명령어를 사용해서 텍스트에 삽입될 수 있다. 명령어에 인자는 물론 좀더 크거나 작은 단위로 변경될 수 있다. 즉, 0.1cm, 2.5cm, 1.3in, 등등. 명령어 \texttt{$\backslash$vspace\{1.1cm\}} 도\\
\vspace{1.1cm}


\noindent 비슷한 방식으로 동작한다. \LaTeX 은 다른 선택옵션 몇가지와 더불어, \texttt{cm}, \texttt{mm}, \texttt{in}, 특정한 길이 단위도 받아들인다.

\subsection{텍스트 (Text)}

명령어를 사용해서 텍스트를 조작할 수도 있다. 예를 들어, \texttt{$\backslash$emph} 명령어를 사용해서 텍스트를 \emph{강조} (이태릭체) 한다.  {\em 텍스트를 이태릭체로 작성하는 방식이 몇가지 있다.} \{ 중괄호 \} 를 사용해서 명령어가 먹히는 곳을 잡아낸다. 이태릭체와 유사하게 텍스트를 {\bfseries 몇가지 방식으로} \textbf{굵은 글씨로} 표현할 수 있거나 텍스트에 {\color{red} 색깔}을 입힐 수도 있다. 
\definecolor{myRed}{rgb}{.7,.2,.1} % RGB 시스템으로, 1번째 숫자는 빨강, 두번째 숫자는 녹색, 세번째 숫자는 청색을 표현.
{\color{myRed} 이 색이 ``내가 정의한 색이다''}. % 색깔을 입힐 때는 ``color'' 팩키지가 포함되어야만 된다.
또한, \texttt{타자수처럼 글자를 타이핑한다}. \\

쉬프트(shift)-아포스트로피 키 조합을 사용해서 "큰 따옴표" 표시하면 \LaTeX 에서 미려하지 못하다. 대신에 왼쪽 큰 따옴표로 ``1 번'' 키옆에 아포스트로피를 두번, 오른쪽 큰 따옴표로 따옴표를 조합해서 사용한다. 즉, ``왼쪽 큰 따옴표와 오른쪽 큰 따옴표" .\\

텍스트 크기도 {\tiny 매우 작게}(tiny), {\scriptsize 스크립트 크기}(scriptsize), {\footnotesize 주석 크기}(footnotesize), {\small 작게}(small), {\large 크게}(large), {\Large 아주 크게}(Large), {\LARGE 매우 크게}(LARGE) 등등 다양한 크기로 표현 가능하다. \\

\subsection{매크로 (Macros)}

일반적으로, 이번 절에서 사용된 명령어는 \textbf{매크로(Macros)}를 사용하면 찾을 수 있다. 예를 들어, TeXShop 프로그램에서 원하는 폰트 크기를 얻으려면, Macros $>$ Text Styles $>$ size 로 들어간다. 인용(quotation)도 시도해 보라 (Macros $>$ Insertions $>$ quotation ):

\begin{quotation} %\em % perhaps you want it italicized?
보통 문단에 포함하고 싶지 않은 매우 긴 인용될 수 있다. 그래서, 다른 한편으로 별도로 독립시켜서 보통 문단보다 좀더 적을 폭을 갖게 만든다. 만약 원한다면, 인용 시작지점에 $\backslash$em 을 사용해서 이러한 특별한 텍스트 전체를 쉽게 이태릭체로 만들 수 있다. 
\end{quotation}

\subsection{글머리표 (Lists)}

앞선 하위 절은 다음과 같다... 
\begin{itemize}
\item 공백 (Spacing)
\item 텍스트 (Text)
\item 매크로 (Macros)
\end{itemize}

글머리표는 \texttt{itemize} 환경을 사용하거나, 만약 숫자형 글머리표를 원한다면, \texttt{enumerate} 를 사용한다:
\begin{enumerate}
\item 공백 (Spacing)
\item 텍스트 (Text)
\item 매크로 (Macros)
\end{enumerate}

하지만, 숫자는 하위 절과 매칭되어야 되고, 이 작업을 수행하는 선택옵션도 물론 있다....

\begin{enumerate}
\item[\ref{spacing}] 공백 (Spacing) % ref{} will be explained in the Tables section.
\item[1.5] 텍스트 (Text)
\item[1.6] 매크로 (Macros)
\end{enumerate}

\subsection{특수 문자 (Special characters)}

\LaTeX 코드는 특수 문자를 많이 사용한다. 이것이 의미하는 바는 텍스트에 이러한 특수 문자를 넣으려면, \LaTeX 코드의 통상적인 목적에서 문자를 \emph{나오게}(escape) 만들어야 된다.
예를 들어, 다음 문자를 나타내려면 명령어 각각은 역슬래쉬를 앞에 두는 것이 필요하다:  \#, \$, \{, \}, \&, \%, \_. $\backslash$ 와 $\sim$ 은 약간 더 호들갑을 떨어야 된다. 그리스 문자와 기호는 ~\ref{math} 절에서 소개된다.

\subsection{글자 그대로 (verbatim)}
명시적으로 텍스트를 정말 찍고자 한다면, 글자 그대로 (verbatim) 기능을 사용한다:

\begin{verbatim}
만약 \LaTeX 파일로 보고자 한다면, 이 모든 것이 정말 그대로 \emph{나타날 것이다}. % 와 주석(comment)도 글자 그대로 (verbatim)에서는 동작하지 않는다...
\end{verbatim}

\section{탭 (Tabbing)}
\label{tabbing}

\subsection{태빙 기초(Basic tabbing)}

indent 명령어를 사용해서 텍스트를 들여쓰기한다: \\
\indent\indent\indent 이것이 들여쓰기한 텍스트다:\\

좀더 중요하게, 사용자 정의 태빙도 만들 수 있다. 예를 들어, 

\begin{tabbing}
첫번째 탭 \= 두번째 탭 \= 디폴트 기본설정으로 탭 사이에 \= 여분 공간은 없다. \\
1 \> 2 \> 3 \> 4 \\
하나 \> \> 셋 \> 넷
\end{tabbing}

세번째 줄 두번째 열처럼, 셀은 공백이 될 수도 있다. \\

\texttt{$\backslash$hspace\{0.2cm\}} 명령어를 사용해서, 좀더 많은 공백을 넣을 수 있다...

\begin{tabbing}
첫번째 탭\hspace{0.2cm} \= 두번째 탭\hspace{0.2cm} \= 이제 탭 사이에 \hspace{0.2cm} \= 추가 공백이 생겼다. \\
1 \> 2 \> 3 \> 4 \\
하나 \> \> 셋 \> 넷
\end{tabbing}

\subsection{즉흥 탭 (Impromptu tabs)}
\label{impromptuTabs}

새로운 탭을 중도에 생성할 수도 있다...

\begin{tabbing}
첫번째 탭\hspace{0.2cm} \= 두번째 탭\hspace{0.2cm} \= 이제 탭 사이에 \hspace{0.2cm} \= 추가 공백이 있다. \\
1 \> 2 \> 3 \> 4 \= 4.5 \\
하나 \> \> 셋 \> 넷 \> 이것은 어디서 시작할까요?
\end{tabbing}

그렇게 잘 동작하지 않는다. \texttt{$\backslash$hspace} 을 사용해서 수정한다...

\begin{tabbing}
첫번째 탭\hspace{0.2cm} \= 두번째 탭\hspace{0.2cm} \= 이제 탭 사이에 \hspace{0.2cm} \= 추가 공백이 있다. \\ [2ex]  % [2ex] is used to make a little extra space (2 can be varied)
1 \> 2 \> 3 \> 4\hspace{1.0cm} \= 4.5 \\
하나 \> \> 셋 \> 넷 \> 이것은 어디서 시작할까요?
\end{tabbing}

\subsection{태빙 예제 (Tabbing example)}

태빙은 흥미로운 환경이다. 좀더 만만찮은 태빙 생성은 (\LaTeX 를 다소 난잡하게 만든다) ...
\texttt{$\backslash$hspace} 명령어는 음수 인자를 받아들일 수 있다. 그렇지 않다면, 다음 예제의 구성요소는 허락되지 않는다.

\begin{tabbing}
\bfseries 테스트 명칭\hspace{0.17cm}
	\= \bfseries 기술 \hspace{-2.0cm}
	\= \hspace{4.0cm}
	\=\hspace{3.5cm}
	\= \bfseries 전체 시험 횟수 \\[2ex] % [2ex] is used to make a little extra space
고정 크기 
	\> 데이터 수집 시에
	\>	\>	
	\> $n_f(\alpha, \beta, \delta, \sigma^2)$  \\
\>
	\> 만약 $|Z_k| \geq 1.96$,
	\> 정지하고, $H_0$ 를 기각.
	\> \\
\>
	\> 그렇지 않으면,
	\> 정지하고, $H_0$ 를기각하지 않는다. 
	\> \\ [3ex]
포콕 (Pocock)
	\> $k=1,...,K-1$ 집단 다음에, 
	\>	\>
	\> $n_fR_P(K,\alpha,\beta)$ \\
\>
	\> 만약 $|Z_k| \geq C_P(K,\alpha)$,
	\> 멈추고, $H_0$ 를 기각.
	\> \\
\>
	\> 그렇지 않다면, 
	\> 테스트를 계속한다.
	\> \\
\>
	$K$ 집단 후에 (마지막 집단)
	\>	\>	\> \\
\>
	\> 만약 $|Z_K| \geq C_P(K,\alpha)$ 이면,
	\> 멈추고, $H_0$ 를 기각.
	\> \\
\>
	\> 그렇지 않다면,
	\> 멈추고, $H_0$ 를 기각하지 않는다.
	\> \\
\end{tabbing}


\section{표 (Tables)}

\subsection{표 기초(Basic tables)}

기본 표는... \\

\begin{tabular}{l c r} % lcr 은 첫번째 열은 왼쪽 정렬, 두번째 열은 중간 정렬, 세번째 열은 오른쪽 정렬.
	왼쪽 정렬 & 중간 정렬 & 오른쪽 정렬 \\ % 앰퍼샌드(&) 가 다음 열이 시작되는 위치를 지정한다.
	1     & 2           & 3  \\
\end{tabular}

표를 가운데 정렬하려면, 표 주위에 가운데 정렬 환경을 생성한다:

\begin{center} % 표 가운데 정렬 시작
\begin{tabular}{l  rrrr} % 정렬 문자 사이에 공백을 넣어도 상관없다.
  \hline % 수평선을 여기 추가한다
 		           & Estimate & Std. Error & t value & Pr($>$$|$t$|$) \\
  \hline
(Intercept) & -0.2852   & 0.8434     & -0.34    & 0.7452 \\
x                & 0.4192    & 0.1499     & 2.80     & 0.0266 \\
   \hline
\end{tabular}
\end{center} % 가운데 정렬 멈춤

아마도, 수직 분할선을 추가하고 싶을 수도 있다 (원할 경우, 수직 분할선을 더 추가할 수도 있다)...

\begin{center}
\begin{tabular}{l | rrrr} % 열들 사이에 수직분할선을 표기하는데, 수직 막대, 'l' 를 사용한다.
% 만약 원한다면 다수 수직막대를 넣을 수 있다 (심지어 이중 수직막대도 가능하다)
   \hline
   \hline % 이중 수평선도 허락된다.
 & Estimate & Std. Error & t value & Pr($>$$|$t$|$) \\
  \hline
(Intercept) & -0.2852 & 0.8434 & -0.34 & 0.7452 \\
  x & 0.4192 & 0.1499 & 2.80 & 0.0266 \\
   \hline
   \hline
\end{tabular}
\end{center}

또다른 표는 다음과 같다...
\begin{center}
\begin{tabular}{lp{7.5cm}r}
\hline
왼쪽 & 왼쪽 정렬된다. & 오른쪽 \\
\hline
1 & 만약 한 칼럼에 텍스트가 너무 길면, 정렬에 사용되는 \texttt{l}, \texttt{c}, \texttt{r} 대신에 \texttt{$\backslash$p\{7.5cm\}} 혹은 이에 상응하는 것을 사용해서 문단이 멋진 형태로 표에 작성되도록 한다. 만약 칼럼 폭을 자유자재로 제어하고자 하면, 매우 편리한 기능이다.& 3 \\
\hline
\end{tabular}
\end{center}

\subsection{캡션과 참조 (Captions and referencing)}

표에 캡션을 원합니까? 표 환경을 사용하세요. 이것을 떠다니는 표(floating table)라고 부른다... 만약 주의깊게 제어하지 않는다면, 그리고 때로는 제어할 때조차도 여전히, 페이지 여기저리를 ``떠다닌다(float)''.   

\begin{table}[h] % [h] 는 "여기에 표를 넣어라"를 의미한다... 다른 선택옵션은 다음과 같다.
			% t = 상단(top)
			% b = 하단(bottom)
			% p = 페이지(page)
			% 상기 선택옵션이 또한 모든 것이자 궁극적인 해답은 아니다... 때때로 표는 작성자가 원치 않는 곳에 가있고, 제어하기도 힘들다.
			% 다수 선택옵션을 나타내는 것도 허용된다, 즉, [h] 대신에 [hbt].
\begin{center}
\begin{tabular}{l  rrrr}
  \hline
 & Estimate & Std. Error & t value & Pr($>$$|$t$|$) \\
  \hline
(Intercept) & -0.2852 & 0.8434 & -0.34 & 0.7452 \\
  x & 0.4192 & 0.1499 & 2.80 & 0.0266 \\
   \hline
\end{tabular}
\end{center}
\caption{이것이 캡션(caption)이다.}
\end{table}

\emph{자동으로} 참조를 표(차후에 나올 그림)에 빌드할 수도 있다. 예를 들어, 다음 표는 표~\ref{multRegression}. 만약 표 번호가 변경되면, .tex 파일을 두번 컴파일한 뒤에 표번호가 자동으로 갱신된다.

왜 두번 컴파일해야 할까요? \LaTeX 이 컴파일할 때 (모두 무시하는 파일들... 컴파일할 때 생성되는 파일 중 하나로부터), 참조(reference)를 불러 읽어온다. 하지만, 불러 읽어오는 파일은  \emph{이전} 컴파일 결과로부터 만들어졌다. 그래서, 만약 한번만 컴파일하면, 읽어오는 파일이 갱신이 되지 않을 수도 있다. (이제 이해되시죠?)

참조에 대한 추가적인 설명은 \texttt{latex-intro-kr.tex} 원본 파일을 참조한다.


% \ref{} 명령어는 특정 \label{} 명령어를 참조한다. 표참조 사이 공백에 ~ 기호를 넣는다. 	``표'' 다음에 곧바로 행바꿈을 LaTeX이 못하게 하는 역할을 한다. (이렇게 하면 약간 더 근사하게 보인다)

% 참조는 실제로 다양한 방식으로 사용된다. 예를 들어, tabbing 절을 참조하려면, \ref{tabbing} 을 사용한다... \label{tabbing} 이 이미 해당 절 시작지점에 위치해야만 된다. 그래서 절 번호가 참조될 수 있다.

% 다시 한번... 만약 참조(reference) 기능을 사용하려면, 항상 LaTeX 문서를 두번 컴파일한다, 즉, 작업결과를 전송, 제출, 출력하기 전에 두번 컴파일한다.

\begin{table}[ht]
\begin{center}
\begin{tabular}{rrrrr}
  \hline
 & Estimate & Std. Error & t value & Pr($>$$|$t$|$) \\
  \hline
(Intercept) & -0.5758 & 1.4528 & -0.40 & 0.7056 \\
  x & 0.3775 & 0.1971 & 1.92 & 0.1039 \\
  z & 1.4042 & 1.7357 & 0.81 & 0.4494 \\
   \hline
\end{tabular}
\end{center}
\caption{ $x$ 혹은 $z$ 어느 변수도 통계적으로 유의하지 않은 것으로 밝혀졌다.}
\label{multRegression} % 해당 표에 대한 라벨.
% 만약 라벨이 동작하려면, 캡션이 있어야만 된다 (그렇지 않다면, 참조할 번호가 없게 된다).
\end{table}

\subsection{\texttt{array} 환경}

\ref{math}~절에서 보았듯이, 통상 방정식에 사용된다는 것만 제외하면, \texttt{array} 환경은 \texttt{tabular} 환경과 매우 유사하다. 

\subsection{R 팩키지, \texttt{xtable}}

R 산출물을 \LaTeX에 넣으려는 R 사용자에게 있어 \texttt{xtable} 팩키지는 매우 유용하다:

\begin{verbatim}
> library(xtable) # install.packages('xtable') 명령어를 사용해서 팩키지를 다운로드한다.
> x <- 1:9
> z <- rnorm(9)
> y <- x/7 + z*2 + rnorm(9)
> xtable(summary(lm(y ~ x+z)))
[... LaTeX 으로 복사해서 붙여넣기 할 수 잇는 산출물이 많다 ...]
\end{verbatim}

R에서 직접 복사해서/붙여넣은 산출물 표: 

% xtable 1.5-4 팩키지로 R 2.8.1 에서 생성한 LaTeX 표.
% Sat Apr 18 14:13:39 2009
\begin{table}[ht]
\begin{center}
\begin{tabular}{rrrrr}
  \hline
 & Estimate & Std. Error & t value & Pr($>$$|$t$|$) \\
  \hline
(Intercept) & -0.1563 & 0.6243 & -0.25 & 0.8107 \\
  x & 0.1094 & 0.1145 & 0.96 & 0.3760 \\
  z & 2.6170 & 0.4308 & 6.08 & 0.0009 \\
   \hline
\end{tabular}
\end{center}
\end{table}

이뿐 아니라, 행렬, 데이터프레임, 다른 R 객체에도 사용될 수 있다.

\section{그림 (Figures)}

\subsection{그림 기초 (Basic figures)}

\includegraphics{lower40} % ``lower40'' 파일 삽입
			% (본 .tex 작업 파일과 동일한 디렉토리에 있음)
			% 확장자 필요없음!

\texttt{$\backslash$includegraphics} 명령어를 사용해서 기본적 그림을 삽입할 수 있음. 
선택옵션 \texttt{space} 인자를 사용해서 크기도 제어할 수 있음.

\includegraphics[height=1.0in]{lower40} % space 선택옵션 사용: [height=1.0in]

표를 가운데 정렬한 것과 동일한 방식으로 그림도 쉽게 가운데 정렬할 수 있다:

\begin{center}
\includegraphics[height=1.0in]{figures/lower82/lower82}
\end{center}

\subsection{캡션과 참조하기 (Captions and referencing)}

표와 마찬가지로, 그림은 ``떠다닐(floated)'' 수 있고, 캡션/라벨을 갖을 수 있다.
Templates 도구로 그래픽 작업을 좀더 멋지게 수행할 수 있다. 그림~\ref{figureTemplate} 참조. \LaTeX ``Float Figure Template'' 에는 추가할 필요가 있는 space 선택옵션이 포함되어 있지 않음에 주목한다.\\

\begin{figure}[htbp]
	% 떠다니는 표와 마찬가지로, 그림은 페이지 여기저기를 떠다니고, [htbp]를 통해 일부 제어기능을 제공한다.
   \centering % 그림을 가운데 정렬하는 한 방법
   \includegraphics[height=2.0in]{figures/figureTemplate}
   	% 디렉토리에 위치한 그림을 참조하는 것에 주목한다.
   	% 주요 작업 디렉토리를 난잡하게 만들지 않도록 주의깊이 파일을 관리하는 똑똑한 방식이다.
   \caption{그림 템플릿을 찾는 장소.}
   \label{figureTemplate}
\end{figure}

\subsection{체계적으로 정리하기 (Keeping organized)}

그림을 별도 디렉토리에 정리할 것을 강력 추천한다.
그림~\ref{messyFolder} 처럼 이미지 파일로 주 작업 디렉토리가 어수선하게 되는 것을 방지한다. 
그림~\ref{cleanFolder}에 문서 그림에 대한 더 좋게 체계적으로 정리된 구조가 나와 있다.\\

\begin{figure}[htbp]
   \centering
   \includegraphics[height=3.0in]{figures/messyFolder}
   \caption{이렇게 하지 말라. ``slideshow'' 보다는 좀더 주의깊게 파일명을 짓는다. ``slideshow'' 는 특정되지 않아 향후 혼란의 여지가 많다.}
   \label{messyFolder}
\end{figure}
\begin{figure}[htbp]
   \centering
   \includegraphics[height=1.8in]{figures/cleanFolder}
   \caption{이와 같이 파일을 체계적으로 정리하라.}
   \label{cleanFolder}
\end{figure}


\section{수학 (Math)}
\label{math}

\subsection{텍스트 내부 수학 (Math in text)}

\LaTeX 을 사용하면, $\alpha$, $\zeta$, $\mu$, 등 그리스 문자를 텍스트 내부에 추가하기 쉽다. 동일한 방식으로, 방정식도 또한 쉽게 추가될 수 있다: $y=x^3$, $\sum z^j$, $x_1+\cdots+x_n$.

\begin{center}
\includegraphics[height=1.5cm]{figures/mathInText}
\end{center}

$\alpha$, $\zeta$, $y=\sqrt{x}\log(x)$ 같은 그리스 문자와 수학 표현식은 달러 기호 두개를 사용해서 쉽게 삽입할 수 있다: 각 수학 표현식과 기호에 대해 어떤 명령어가 대응되는지 단지 기억력과 관련되 문제다. 예를 들어, $\alpha$ 는 \texttt{\$$\backslash$alpha\$} 명령어로 생성된다. $\alpha$가 생성되는 방식에 기반해서, $\beta$ 는 어떻게 생성할까요?

LaTeX 과 Matrix 패널에는 상당히 많은 수의 기호와 문자 등이 포함되어 있다.  맥 TeXShop, 윈도우 TeXWork 에서 \texttt{alt-command-[dash/underscore key]} 혹은 \texttt{alt-command-[+/= key]} 단축키를 통해서, 혹은 메뉴를 통해 해당 패널(TeXShop ``Window''메뉴)로 접근할 수 있다. 생성할 수 있는 일부 문자와 기호가 다음에 나와 있다...\\

$\hbar\imath\jmath\ell\Re\Im\emptyset\infty\partial\nabla\triangle\forall\exists\nexists\top\bot\dag\ddag\sum\prod\int\oint\bigcap\cap\bigcup\cup\biguplus\bigoplus\bigotimes\bigodot\hat{a}\bar{a}\tilde{a}$ \\

$\alpha\beta\gamma\delta\epsilon\varepsilon\zeta\eta\theta\iota\kappa\lambda\mu\nu\xi\pi\varpi\varrho
\sigma\varsigma\tau\upsilon\phi\varphi\chi\psi\omega$ \\

이용가능한 기호는 어마어마하다. 만약 특정 기호가 필요하다면, 아마도 \LaTeX 에 존재할 것이다.\\

% LaTeX 패멀에서 찾을 수 없다면, 다음 PDF 파일에서 원하는 것을 찾아보세요.
%	www.ctan.org/tex-archive/info/symbols/comprehensive/symbols-a4.pdf

수학 표현식을 만들어 내는 방식이 엄청나게 많이 있다... 

\begin{eqnarray*} % 잠시동안 신규 eqnarray 환경에서 작업한다...
\sqrt{2}, \quad \frac{5}{2+3}=1, \quad \left(\frac{5}{2+3}\right), \quad 2^10 \neq 2^{10} = 1024, \quad x_1 = 3 \\
\bar{x}, \quad 3 \geq x, \quad \lim_{x\to0}\left( \frac{\sin(x)}{x} \right) \to 1, \quad \frac{\sin(x)}{x}\stackrel{x\to0}{\to} 1
% \quad 공백을 약간 삽입한다.
% \frac 는 인자가 두개 있다: 분모와 분자
% \left( 와 \right) 는 자동으로 표현식 내부 크기에 맞춰 괄호를 생성한다. 즉, (\frac{5}{2+3}) 표현식은 그다지 멋져 보이지 않는다.
% ^ 기호를 사용해서 윗첨자를 만들고, 비슷한 방식으로 아랫첨자에 대해서 _ 기호를 사용한다.
% 윗첨자와 아래첨자를 넣을 때, 단일 문자보다 많으면 괄호를 사용해서 해당 문자를 넣는다. 즉 x_{윗첨자} or x^{아랫첨자}.
\end{eqnarray*}

\subsection{방정식 환경과 참조 (Equation environment and referencing)}

방정식 환경을 사용해서 방정식을 그 자체 라인으로 넣을 수 있다:

Equations can also be put on their own line using the equation environment:
\begin{eqnarray}
A_{b_{ik}} % 줄바꿈은 문제가 되지 않는다.
		% 단일 문자보다 많은 첨자를 넣는다면, 중괄호 사용을 확실히 한다 (그러지 않으면, 동작하지 않게 된다...)
		% 상기 방정식 2^10 \neq 2^{10} = 1024 을 참조한다.
	= \sum_{l=1}^{k}\sum_{j=1}^{i} \gamma^{\alpha_{b_{jl}}}
\label{Abi}
\end{eqnarray}

그림과 표와 마찬가지로, 방정식도 방정식~\ref{Abi} 처럼  참조할 수 있다. \\

만약 방정식에 숫자를 배정하고 싶지 않다면, \texttt{eqnarray$^*$} 환경을 사용한다:

\begin{eqnarray*}
A_{b_{ik}} = \sum_{l=1}^{k}\sum_{j=1}^{i} \gamma^{\alpha_{b_{jl}}}
\end{eqnarray*}

다음에 방정식~\ref{powerSeries} 사례가 하나더 나와 있다...

\begin{eqnarray}
\sum_{k=0}^{\infty}0.5^k = \frac{1}{1-0.5} = 2
\label{powerSeries}
\end{eqnarray}


\subsection{줄 맞추기 (Aligning)}

방정식이 여러 줄에 걸쳐있고, 줄 맞추기가 필요하다면, 앰퍼샌드(\&)를 두개를 사용한다:

\begin{eqnarray*}
y &=& (x-b)^2 + a \\ % 새줄로 개행하려면, 이중 역슬래쉬가 필요하다. 
&=& x^2 - 2bx + b^2 + a % 등치 부호 앞뒤에  & 기호가 각 줄 맞추기 기능을 한다.
\end{eqnarray*}

만약 앰퍼샌드를 사용하지 않으면, 줄맞추기가 보통 엉망이 된다.

\subsection{배열 (Arrays)}

Matrix 패널을 사용해서 항상 배열을 쉽게 생성한다:

\begin{eqnarray*}
\left(
	\begin{array}{ccc}
	\sigma_1^2 & \sigma_{1,2} & \sigma_{1,3} \\
	\sigma_{2,1} & \sigma_{2}^2 & \sigma_{2,3} \\
	\sigma_{3,1} & \sigma_{3,2} & \sigma_{3}^2
	\end{array}
\right)
\end{eqnarray*}

수학 기호를 삽입하기 더 쉽다는 점을 제외하면, 배열 생성은 본질적으로 표와 동일하다.

\subsection{ \texttt{amsmath} 팩키지 사용 시 몇가지 장점}

\texttt{amsmath} 팩키지는 \LaTeX 템플릿에 없다. 하지만, 있으면 매우 편리하다.
만약 좀더 긴 방정식이 있고, 한 줄에만 숫자를 넣고 싶다면, \texttt{$\backslash$notag} 을 사용한다:

\begin{eqnarray}
&&Cov\left( \left(\bar{X}_{A}^{(k_1)} - \bar{X}_{B}^{(k_1)}\right)\sqrt{I_{k_1}}, \left(\bar{X}_{A}^{(k_2)} - \bar{X}_{B}^{(k_2)}\right)\sqrt{I_{k_2}} \right) \notag \\
&&\quad= Cov\left( \bar{X}_{A}^{(k_1)} - \bar{X}_{B}^{(k_1)}, \bar{X}_{A}^{(k_2)} - \bar{X}_{B}^{(k_2)}\right) \sqrt{I_{k_1}I_{k_2}} \notag \\ % \quad 는 공백을 일부 추가한다
&&\quad= Cov\left( \bar{X}_{A}^{(k_1)} - \bar{X}_{B}^{(k_1)}, \bar{X}_{A}^{(k_2)} - \bar{X}_{B}^{(k_2)}\right) \sqrt{I_{k_1}I_{k_2}}
\end{eqnarray}

\texttt{amsmath} 팩키지가 이러한 명령어를 사용하는데 필요하다. \texttt{$\backslash$text} 를 사용해서 방정식에 텍스트를 추가할 때도 이 팩키지가 필요하다:

\begin{eqnarray*}
\bar{x} = \sum_{i=1}^{n} x_i \quad \text{ 그리고 } \quad \hat{\sigma} = \sqrt{\frac{1}{n-1}\sum_{i=1}^n(x_i-\bar{x})^2}
\end{eqnarray*}

\texttt{$\backslash$text}를 갖는 \texttt{eqnarray$^*$} 의 또다른 사례:

\begin{eqnarray*}
\text{추정 시간} = \frac{\text{여행 거리}}{\text{자동차 속도}} + \text{지연 시간}
\end{eqnarray*}

\section{실습 (Practice)}

새로운 문서를 생성해서 항목 3개를 만드세요. 새로운 문서에 \texttt{$\backslash$title} 와 \texttt{$\backslash$author} 정보를 확실히 갱신하세요.

\subsection{ 시도해 보자 \#1}

\texttt{tabular} 환경을 사용해서 그림~\ref{tryIt1}에 나온 출력을 만들어 본다.

\begin{figure}[htbp]
   \centering
   \includegraphics[height=0.8in]{tryIt/tryIt1}
   \caption{시도해 보자 \#1.}
   \label{tryIt1}
\end{figure}

\subsection{시도해 보자 \#2}

다음 이미지를 높이 2 센티미터(0.8 인치), 가운데 정렬, 캡션 추가, 참조 추가해서 변형하세요. 그림을 참조하는 문장도 작성하고 \LaTeX 문서를 두번 컴파일 해서 참조가 정상 동작되도록 한다. (만약 ``Float Figure Template'' 을 사용하면, 높이 선택옵션을 추가하는 것을 확실히 하라... 대안으로, 템플릿으로 이전 \LaTeX 예제를 사용할 수도 있다.)\\

\includegraphics{figures/lower82/lower82}

\subsection{시도해 보자 \#3}

\texttt{eqnarray$^*$} 환경을 사용해서, 그림~\ref{tryIt3}에 나온 방정식을 만들어 보세요.

\begin{figure}[htbp]
   \centering
   \includegraphics[height=0.5in]{tryIt/tryIt3}
   \caption{시도해 보자 \#3.}
   \label{tryIt3}
\end{figure}


\section{참고문헌 관련}

점 패턴(point pattern)이 점과정(point process) 구현\citep{daley}으로 기술되고, 점 패턴에 대한 몇가지 일차원 거리함수가 \citet{victor}에 기술되어 있다.

\bibliographystyle{biblio/simpleStyle}
\bibliography{biblio/bibDB}


\end{document}  