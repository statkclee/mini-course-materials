%\documentclass[slidestop,compress,mathserif]{beamer}
\documentclass[10pt]{beamer}
\usetheme{Frankfurt}
\usecolortheme{seagull}

% Load packages
\usepackage{alltt}
\usepackage{verbatim}
\usepackage{geometry}
\usepackage{graphicx}
\usepackage{amssymb}
\usepackage{amsmath}
\usepackage{epstopdf}
\usepackage{feynmp}
\usepackage{authblk}
\usepackage[hangul]{kotex}
\usepackage[T1]{fontenc}
\usepackage[utf8]{inputenc}
%\usepackage{setspace}

%\usepackage{pgfpages}
%\pgfpagesuselayout{4 on 1}[letterpaper,landscape,border shrink=5mm]
    \usepackage{kotex-logo}
    \usepackage{iftex}
    \ifPDFTeX
      \usepackage{dhucs-nanumfont}
      \hypersetup{pdftex,unicode,bookmarks}
      \input glyphtounicode
      \pdfgentounicode=1
    \else\ifXeTeX
      \setmainhangulfont[Ligatures=TeX]{HCR Batang LVT}
      \setsanshangulfont[Ligatures=TeX]{HCR Dotum LVT}
    \else\ifLuaTeX
      \hypersetup{unicode,bookmarks}
      \setmainhangulfont[Ligatures=TeX]{HCR Batang LVT}
      \setsanshangulfont[Ligatures=TeX]{HCR Dotum LVT}
    \fi\fi\fi

\providecommand\thispdfpagelabel[1]{}

\DeclareGraphicsRule{.tif}{png}{.png}{`convert #1 `dirname #1`/`basename #1 .tif`.png}


% Title 
% Note: [short title]{long title}, [short author(s) name]{long author(s) name}
\title{\LaTeX\ 소개}
\subtitle{공개 통계학 개론 (OpenIntro Stat.) 저작 학습용}

\author[David]{}
%\author{David Diez \inst{1} \and 이광춘(번역) \inst{2}} 
\institute{\inst{1} \href{http://www.openintro.org}{openintro.org} \and \inst{2} \href{http://www.xwmooc.net}{xwMOOC} }
\newcommand{\samelineand}{\qquad}


\date{}

\begin{document}
\definecolor{highlight}{rgb}{.7,.1,.1}
\definecolor{command}{rgb}{.1,.1,.9}
\definecolor{comment}{rgb}{1,0,0}
\definecolor{braces}{rgb}{0,0.5,0}
\newenvironment{act}[1]{{\color{command}#1}}{}


\frame{ \titlepage }

\begin{frame}
  \frametitle{목차}
  \begin{itemize}
  \item \LaTeX\ 인터페이스 소개
  \item 텍스트 작업 
  \item 탭과 표
  \item 그림
%  \item Math and equations
  \end{itemize}
\end{frame}

\part{}

\section[시작하기]{시작하기}

\subsection[시작하기]{시작하기}

\begin{frame} \frametitle{\LaTeX 설치}
{\bf 맥(Mac) 설치. } {\color{highlight}http://www.tug.org/mactex/} 사이트에서 MacTeX을 다운로드한다. MacTeX 설치에 포함된 \LaTeX\은 TeXShop 프로그램으로 접근해 사용한다. \\
\vspace{0.3cm}

{\bf 윈도우(Windows) 설치. } {\color{highlight}http://www.tug.org/protext/}\footnote{또다른 선택옵션: {\color{highlight}http://www.winshell.de/}. (Winshell 과 MikTeX)} 사이트에서 ProTeXt 를 다운로한다. ProTeXt 실치에 포함된 TeXnicCenter 프로그램으로 접근해 사용한다. \\
\vspace{0.3cm}

MacTeX 과 proTeXt 모두 대용량이다: 각각 1.2GB 와 540MB. \\
\vspace{0.3cm}

맥, 윈도우, 리눅스(TexWorks) 인터페이스는 다르지만, \LaTeX ``코드''는 \textbf{세가지 플랫폼} 모두 동작한다.
\end{frame}

\begin{frame} \frametitle{TeXShop 열기}
{\color{highlight}Finder (아이콘: \hspace{1mm}\includegraphics[height=0.28cm]{basicsOfLatex/gettingStarted/finder}\hspace{1mm}) $\to$ Applications $>$ TeX $>$ TeXShop} \\
\begin{center}
\includegraphics[height=1.7in]{basicsOfLatex/gettingStarted/openingTexshop}
\end{center}
\end{frame}

\begin{frame} \frametitle{기본 문서 생성하기}
{\bf 파일 열기. } 만약 아직 문서가 열려있지 않다면, {\color{highlight}File $>$ New} 혹은 \texttt{\color{highlight}command-N} (\texttt{\color{highlight}\emph{apple}-N}) 을 사용해서 새문서를 연다. \\
\vspace{0.7cm}

{\bf 템플릿 (Templates). } LatexTemplate 을 선택한다.
\begin{center}
\includegraphics[height=0.95in]{basicsOfLatex/gettingStarted/chooseTemplate}
\end{center}
\end{frame}


\begin{frame} \frametitle{기본 문서 생성하기}
\texttt{\color{command}$\backslash$title}, \texttt{\color{command}$\backslash$author} 정보를 갱신하고 나서, 마지막 행위에 문장을 타이핑한다. 즉, \texttt{\color{command}$\backslash$end}{\color{braces}\{}document{\color{braces}\}} 바로 위에 타이핑한다. 바탕화면에 디렉토리를 생성하고 나서 작성한 해당 \LaTeX 파일을 그 디렉토리 안에 저장한다.

\begin{center}
\includegraphics[height=2.0in]{basicsOfLatex/gettingStarted/firstDoc}
\end{center}
\end{frame}


\begin{frame} \frametitle{조판(Typesetting) / 컴파일}

윈도우 좌상에 위치한 \texttt{\color{highlight}command-T} 단축키 혹은 {\color{highlight}조판}(Typeset) 버튼을 클릭한다. 조판을 한뒤에 확대하려면 PDF 페이지에서 두번 클릭한다. 세번 클릭, 네번 클릭도 시도해 본다.

\begin{center}
\includegraphics[height=2.0in]{basicsOfLatex/gettingStarted/firstOutput}
\end{center}
\end{frame}


\begin{frame} \frametitle{더 추가된 파일}
조판을 하면 다른 파일이 많이 생성된다.
\begin{center}
\includegraphics[height=1.0in]{basicsOfLatex/gettingStarted/outputFiles}
\end{center}

자동 생성된 파일 각각이 특별한 목적이 있지만, 단지 한 파일 -- 원 \LaTeX\ 파일에 추가해서 -- 만 관심있다: PDF. 더 많은 \LaTeX\ 방법을 사용하면, \LaTeX] 출력 파일 목록이 증가할 것이다... 하지만, 다시 한번, \LaTeX\ 을 사용하는 대부분 경우에 .tex 와 .pdf 파일을 제외하고 파일 대부분은 무시된다. 
\end{frame}

\begin{frame} \frametitle{콘솔(Console)}

코드가 조판될 때, 윈도우 두개가 불쑥 팝업된다. {\color{highlight}콘솔}(console)은 문서를 만들어낼 때 \LaTeX\이 무슨 작업을 수행하는지 일러준다. 만약 오류(혹은 \LaTeX\이 원치않는 것)가 있다면, 콘솔이 일러준다. 만약 오류가 심대하면, \LaTeX\이 컴파일을 중지한다:

\begin{center}
\includegraphics[height=1.1in]{basicsOfLatex/gettingStarted/consoleFail}
\end{center}
항상 문제를 즉시 고치는 것이 좋다 (\includegraphics[height=0.25cm]{basicsOfLatex/gettingStarted/gotoError}를 누른다), 일부 오류는 {\color{highlight}엔터}(enter 혹은 return)키를 치고 무시하고 넘어간다.
\end{frame}


\begin{frame} \frametitle{오류는 피할 수 없다.}
오류가 \LaTeX 에서는 흔하다. 오류 식별을 돕기 위해서, 자주 조판할 것을 추천한다. 몇문장마다 조판해서 출력결과가 예상한 것과 매칭이 되는지 확인한다. \LaTeX 이 문서를 처리하는 동안에 작업을 계속 진행할 수 있다. \\
\vspace{0.3cm}

작업을 하면서 좀더 말이 되는 일만적인 오류...
\begin{itemize}
\item 명령어 철자 오류
\item 특수 문자에서 빠져나오지 않음 (이스케이핑, Escaping)
\item {\color{braces}\{}중괄호{\color{braces}\}} 짝을 맞추지 않음
\item {\color{braces}\$} 기호 짝을 맞추치지 않음
\item 시작 환경과 끝 환경 짝을 맞추지 않음. (즉, \texttt{\color{command}$\backslash$begin\color{braces}\{\color{black}document\color{braces}\}} 와 \texttt{\color{command}$\backslash$end\color{braces}\{\color{black}document\color{braces}\}})
\end{itemize}
\end{frame}


\begin{frame} \frametitle{주석}
방금 생성한 기본 파일로 돌아가자. 빨간색(윈도우에서 회색)으로 된 것은 무엇일까? 주석으로 \LaTeX 에서 무시된다. 주석은 퍼센트 기호로 작성된다 : {\color{comment}\%}.

%\begin{tabbing} \small
%\hspace{1cm} \= text text text text text text text text text \\
%	\> text text text text text text text text text \\
%	\> {\color{comment}\% 929j(*\&@\#wfwj9f8uv9 ... this would be ignored} \\
%	\> {\color{comment}\% \hspace{0.5cm} even though it is atrocious and obnoxious} \\
%	\> text text text text text text text text text \\
%	\> text text text text text text text text text
%\end{tabbing}
\begin{center}
\includegraphics[height=0.6in]{basicsOfLatex/gettingStarted/commenting}
\end{center}

\emph{해당 라인만} {\color{comment}\%} 다음에 텍스트는 \LaTeX에서 무시된다.

\end{frame}

%text text text text text text text text text text text text text text text
%text text text text text text text text text text text text text text text
% 92rj9\*U(Q\*U\@\#\#(Wqej ... this would be ignored...
%       even though it is atrocious and obnoxious
%text text text text text text text text text text text text text text text
%text text text text text text text text text text text text text text text


\begin{frame} \frametitle{두 부분으로 된 화면}
TeXShop에는 {\color{highlight}Templates} 오른편에 있는 아이콘 \includegraphics[height=0.25cm]{basicsOfLatex/gettingStarted/dualScreenIcon} \hspace{0.3cm}을 클릭해서 화면을 두 부분으로 분할하는 선택옵션도 있다.

\begin{center}
\includegraphics[height=1.5in]{basicsOfLatex/gettingStarted/dualScreens}
\end{center}
장문의 문서를 작성하거나 편집할 때 매우 유용할 수 있다.
\end{frame}

\begin{frame} \frametitle{예제 템플릿}
로컬 컴퓨터에 다음 압축 디렉토리\footnote{URL로 연결되지 않는다면, {\color{highlight}http://scc.stat.ucla.edu/}, mini-course에서 압축파일을 찾는다.}를 다운로드한다.

\begin{center}
{\color{highlight} http://www.stat.ucla.edu/$\sim$david/latexTemp.zip}
\end{center}

압축파일을 푼다 (맥에서 두번 클릭). {\color{highlight}latexTemp} 디렉토리의 주요 콘텐츠는 다음과 같다: 
\vspace{0.0mm}
\begin{itemize}
\item 예제 문서 ({\color{highlight}latexTemp.tex})
\vspace{-0.3cm}
\item 이 발표자료 PDF 파일
\vspace{-0.3cm}
\item {\color{highlight}figures} 디렉토리
\vspace{-0.3cm}
\item UCLA 논문 템플릿 파일: {\color{highlight}uclathes-1.2} 와 {\color{highlight}uclathesUse}
\end{itemize}

{\color{highlight}latexTemp.tex} 파일은 이 발표자료 내용 대부분을 반영하여 작성했고, 예제와 더블어 추가적인 주석으로 구성되었다. 이 파일을 이제 열어본다.
\end{frame}


%%%%%%%%%%%%%%%%%%%%%%
%\part{}

\section[일반]{일반}

\subsection[표제 절]{표제 절}
\begin{frame} \frametitle{문서 유형}
모든 \LaTeX 문서의 첫번째 명령어는 {\color{command}$\backslash$documentclass}가 된다. 기본적으로 어떤 유형의 문서를 만들 것인지 지정한다.
\begin{center}
\includegraphics[height=0.2in]{basicsOfLatex/general/documentClass}
\end{center}
디폴트 기본설정 \texttt{\color{highlight}article} 클래스다.
\vspace{5mm}\\
다른 클래스: \texttt{\color{highlight}letter}, \texttt{\color{highlight}beamer} (발표), \texttt{\color{highlight}book}. 다른 클래스에 대해서는 예제와 도움말을 온라인에서 찾을 수 있다.
\vspace{5mm} \\
만약 필요하면, \texttt{\color{highlight}[11pt]} 을 변경해서 문서 전체에 적용될 디폴트 기본설정 폰트 크기를 정한다. 
\end{frame}


\begin{frame} \frametitle{팩키지}
팩키지는 추가 기능을 제공하고 일반적으로 무료다. 항상 문서 시작부분에 적재된다.\\
\begin{figure}[htbp]
   \centering
   \includegraphics[height=0.7in]{basicsOfLatex/general/packages}
\end{figure}
\LaTeX 설치 시점에 대부분 많이 사용되는 패키지가 포함된다. 하지만, 다른 많은 팩키지가 포함되지는 않는다. 필요하면 부가 팩키지를 다운로드해서 설치해야만 된다 (여기서는 다루지 않는다).
\end{frame}


\subsection[개요]{개요}
\begin{frame} \frametitle{절과 하위절}
문서는 종종 절과 하위절로 분해된다. 이러한 위계체계는 \LaTeX에서 자동으로 숫자가 매겨진다.

\begin{figure}[htbp]
   \centering
   \includegraphics[height=1.4in]{basicsOfLatex/general/sectionsSubsections}
\end{figure}
\end{frame}

\subsection[텍스트]{텍스트}
\begin{frame} \frametitle{문단}

한 문단을 끝내고 새로운 문단을 생성하려면, ``{\color{highlight}엔터}(enter)''를 두번 친다. 그러면 줄바꿈이 된다.\\
\vspace{0.5cm}
문단사이에 추가 줄공백을 두려면, \texttt{.tex} 문서에 이중 줄바꿈으로 \texttt{\color{command}$\backslash\backslash$} 명령어를 사용한다. 

\begin{figure}[htbp]
   \centering
   \includegraphics[height=1.4in]{basicsOfLatex/general/paragraphSpacing}
\end{figure}
\end{frame}


\begin{frame} \frametitle{들여쓰기와 공백}
{\bf 들여쓰기. } 들여쓰기 하려면, \texttt{\color{command}$\backslash$indent}을 사용한다. 들여쓰기 하지 않으려면, \texttt{\color{command}$\backslash$indent}을 사용한다. 들여쓰기를 \texttt{\color{command}$\backslash$setlength}\texttt{\color{braces}\{}\texttt{\color{command}$\backslash$parindent}\texttt{\color{braces}\}}\texttt{\color{braces}\{}\texttt{0in}\texttt{\color{braces}\}}을 사용해서 막을 수 있다.\\
\vspace{0.7cm}
{\bf 공백. } 수평 \hspace{1cm} 공백을 만들어 내려면, \texttt{\color{command}$\backslash$hspace}\texttt{\color{braces}\{}\texttt{1cm}\texttt{\color{braces}\}} 명령어를 사용한다. 마찬가지로, \texttt{\color{command}$\backslash$vspace}\texttt{\color{braces}\{}\texttt{0.5cm}\texttt{\color{braces}\}} 명령어를 사용하거나, 줄바꿈(디폴트 기본설정 이상) 다음에 추가 공백을 추가하려면, \texttt{{\color{command}$\backslash\backslash$}[1cm]}을 사용한다.
음수 거리도 사용될 수 있다. 
%\texttt{\color{command}$\backslash$hspace} 와 \texttt{\color{command}$\backslash$vspace}는 일반적으로 텍스트보다는 표에 더 유용하다.
\end{frame}

\begin{frame} \frametitle{폰트 만지작 거리기}
{\bf 강조 (이태릭체). } \texttt{\color{command}$\backslash$emph} 명령어를 사용한다. 즉, 
단어를 \emph{강조}(이태릭체)하는데 \texttt{\color{command}$\backslash$emph}{\color{braces}\{}\texttt{강조\color{braces}\}}를 사용한다.
\\
\vspace{0.7cm}

{\bf 좀더 조작하기. } 
텍스트를 \texttt{\color{command}$\backslash$textbf} 으로 \textbf{굵게} 만들거나 \texttt{\color{braces}$\{$\color{command}$\backslash$color\color{braces}$\{$\color{black}red\color{braces}$\}$\color{black}colored\color{braces}$\}$} 으로 {\color{red} 색깔}을 넣을 수도 있다 ({\color{highlight}color} 팩키지가 필요). \texttt{\color{command}$\backslash$texttt}으로 \texttt{타자기로 친 느낌}을 줄 수도 있다.\\
\vspace{0.7cm}

{\bf 폰트 크기. } 
\texttt{\color{command}$\backslash$tiny}, \texttt{\color{command}$\backslash$scriptsize} 등등을 사용해서, 텍스트를 다양한 크기로 만들 수 있다. {\tiny 매우 작은 크기}, {\scriptsize 스크핍트 크기}, {\footnotesize 주석 크기}, {\small 작은 크기}, {\large 큰 크기}, {\Large 아주 큰 크기}, {\LARGE 매우 큰 크기}.
\end{frame}


\begin{frame} \frametitle{글머리표(Lists)}
글머리표(list)를 다음 명령어로 생성한다.
\begin{center}
\includegraphics[height=0.7in]{basicsOfLatex/general/list}
\end{center}

상기 명령어를 실행하면 다음과 같이 글머리표 목록이 생성된다:
\begin{itemize}
\item Spacing (공백)
\item Text (텍스트)
\item Macros (매크로)
\end{itemize}

{\color{highlight}latex-intro-kr.tex} 문서에 추가 예제가 몇가지 더 있다.
\end{frame}

\begin{frame} \frametitle{매크로(Macros)}
Macros(매크로)를 사용해서 좀더 많은 텍스트 변환을 할 수 있다.
\begin{figure}[htbp]
   \centering
   \includegraphics[height=1.8in]{basicsOfLatex/general/macros}
\end{figure}
텍스트 변환에 Macros(매크로)가 아주 많은 선택옵션을 제공한다.
\end{frame}

\begin{frame} \frametitle{실습}
\textit{기울여쓰기}(italicized)와 \textbf{굵게}(bolded) 표시된 단어와 구(phrase)를 포함한 짧은 문단 두개를 작성하시오.

\vspace{7mm}

두 문단 사이 줄바꿈을 하려면, 무엇을 해야 할까요? 두번째 문단만 들여쓰기를 하려면 어떻게 해야 할까요?

\vspace{7mm}

{\color{highlight}color} 팩키지를 추가하고 텍스트 일부에 색깔을 추가하라. 예를 들어, \texttt{\color{braces}$\{$\color{command}$\backslash$color\color{braces}$\{$\color{black}red\color{braces}$\}$\color{black} 텍스트 일부를 붉은 색으로 표시된다.\color{braces}$\}$} 와 같이 시도해 보세요.
\end{frame}

\section[표 / 탭(tabbing)]{표 / 탭(tabbing)}

\subsection[탭(Tabbing)]{탭(Tabbing)}
\begin{frame} \frametitle{탭(Tabbing)}
다른 텍스트 편집기와 마찬가지로, \LaTeX은 탭(Tabbing)기능을 제공한다. 이 환경은 드물게 사용되는 경향이 있지만(적어도 저자 입장에서), 특정 조건에서는 매우 유용할 수 있다.

\begin{figure}[htbp]
   \centering
   \includegraphics[height=1.3in]{basicsOfLatex/tabTable/tabbingExample}
\end{figure}
탭에 대한 간략한 소개와 몇가지 예제에 대해서 {\color{highlight}latex-intro-kr.tex} 파일을 참조한다.
\end{frame}

\subsection[표 기초]{표 기초}
\begin{frame} \frametitle{표 기초}

표는 \texttt{\color{highlight}tabular} 환경을 사용해서 생성한다. \text{\color{braces}\{}\texttt{lcr}\texttt{\color{braces}\}} 인자는 좌/우/가운데 정렬 기능을 제공한다. 앰퍼샌드({\color{highlight}\&})를 사용해서 다음 칼럼으로 넘어가는 것을 정의한다. 

\begin{figure}[htbp]
   \centering
   \includegraphics[height=0.55in]{basicsOfLatex/tabTable/basicTable}
\end{figure}

\texttt{\color{command}$\backslash\backslash$} 명령어를 사용해서 \LaTeX 으로 하여금 새로운 행을 시작하게 한다.\\
\vspace{0.3cm}
결과는 다음과 같다: \\
\vspace{0.3cm}
\begin{tabular}{lcr} % lcr means make the 1st column left aligned, the 2nd centered, and the 3rd right aligned
	Left & Center & Right \\ % the amperstands (&) define where to start the next column
	1     & 2           & 3  \\
\end{tabular}
\end{frame}


\begin{frame} \frametitle{가운데 정렬과 선(line) 추가}
다음 표를 빌드하면:
\begin{figure}[htbp]
   \centering
   \includegraphics[height=1.0in]{basicsOfLatex/tabTable/centeredTable}
\end{figure}
결과는 다음과 같다: \\
\vspace{0.3cm}
\begin{center} % center the table
\begin{tabular}{l | rrrr} % the second 'l' is a vertical bar
  \hline % add a horizontal line here
 		           & Estimate & Std. Error & t value & Pr($>$$|$t$|$) \\
  \hline
(Intercept) & -0.2852   & 0.8434     & -0.34    & 0.7452 \\
x                & 0.4192    & 0.1499     & 2.80     & 0.0266 \\
   \hline
\end{tabular}
\end{center}
\end{frame}

\subsection[캡션 \& 참조]{캡션 \& 참조}
\begin{frame} \frametitle{캡션을 갖는 떠다니는 표}
캡션 혹은 라벨을 추가하려면, 표가 \emph{떠다녀야}(floated) 된다 (즉, \texttt{\color{highlight}table} 환경에 추가한다). 그리고 나서, \texttt{\color{command}$\backslash$caption} 을 사용할 수 있다:

\begin{figure}[htbp]
   \centering
   \includegraphics[height=0.6in]{basicsOfLatex/tabTable/floatTable}
\end{figure}
\vspace*{-0.3cm}
출력 결과는 다음과 같다:
\begin{table}[htbp]
\begin{center} % center the table
\begin{tabular}{l | rrrr} % the second 'l' is a vertical bar
  \hline % add a horizontal line here
 		           & Estimate & Std. Error & t value & Pr($>$$|$t$|$) \\
  \hline
(Intercept) & -0.2852   & 0.8434     & -0.34    & 0.7452 \\
x                & 0.4192    & 0.1499     & 2.80     & 0.0266 \\
   \hline
\end{tabular}
\end{center}
\caption{회귀분석 요약.}
\end{table}
\vspace*{-0.5cm}
캡션이 \texttt{\color{highlight}article} 문서에 추가되면 표에 번호가 매겨진다.
\end{frame}

\begin{frame} \frametitle{참조}
%Tables can be dynamically referenced (many examples in {\color{highlight}latexTemp.tex}).
%\vspace{5mm} \\

문서를 작성하고 표4를 참조한다고 가정하자. 이틀 뒤에 문서 앞쪽에 또다른 표를 추가한기로 결정했다. 이제 표4는 실질적으로 표5가 되고, 연관된 모든 4는 5로 (5는 6으로 등등) 바꿔야 한다.
\vspace{5mm} \\
대신에, 표 각각을 유일무이한 꼬리표를 붙인다. 숫자 번호가 아닌 꼬리표를 참조한다.

\begin{figure}[htbp]
   \centering
   \includegraphics[height=0.4in]{basicsOfLatex/tabTable/labelTable}
\end{figure}

\texttt{\color{command}$\backslash$label}\texttt{\color{braces}\{}\texttt{regressTable}\texttt{\color{braces}\}} 꼬리표가 표에 붙었고, 이 표를 \LaTeX 코드를 사용해서 참조한다.

\texttt{표$\sim$}\texttt{\color{command}$\backslash$ref}\texttt{\color{braces}\{}\texttt{regressTable}\texttt{\color{braces}\}}
\end{frame}

\subsection[xtable]{xtable}
\begin{frame} \frametitle{R 팩키지 \texttt{xtable} 사용하기} 
R의 내부: \\
\vspace{0.1cm} \small
\hspace{0.3cm}\texttt{> library(xtable) \# install.packages('xtable') 실행이 필요할 수도 있다.} \\
\hspace{0.3cm}\texttt{> x <- 1:9} \\
\hspace{0.3cm}\texttt{> z <- rnorm(9)} \\
\hspace{0.3cm}\texttt{> y <- x$/$7 + z$^*$2 + rnorm(9)} \\
\hspace{0.3cm}\texttt{> xtable(summary(lm(y $\sim$ x$+$z)))} \\
\hspace{0.3cm}\texttt{... \LaTeX 내부로 복사/붙여넣기 할 수 있는 출력결과...} \\
\vspace{0.1cm} \normalsize
R 출력결과를 \LaTeX 내부로 직접 복사해서 붙여넣기 한다:
\begin{table}[ht]
\begin{center}
\begin{tabular}{rrrrr}
  \hline
 & Estimate & Std. Error & t value & Pr($>$$|$t$|$) \\
  \hline
(Intercept) & -0.1563 & 0.6243 & -0.25 & 0.8107 \\
  x & 0.1094 & 0.1145 & 0.96 & 0.3760 \\
  z & 2.6170 & 0.4308 & 6.08 & 0.0009 \\
   \hline
\end{tabular}
\end{center}
\end{table}
\end{frame}

\frame{ \frametitle{실습}
다음 출력결과를 생성하라:
\begin{center}
\begin{tabular}{l rrr}
\hline
& mean & sd & n \\
\hline
S1 & 6.5 & 1.3 & 17 \\
S2 & 12.2 & 1.4 & 25 \\
\hline
\end{tabular}
\end{center}

\texttt{\color{highlight}latex-intro-kr.tex} 파일에 일부 예제를 활용할 수도 있다.
}


\section[그림]{그림}
\subsection[그림]{그림}
\begin{frame} \frametitle{그림 기초}
{\color{command}$\backslash$includegraphics} 명령어를 사용해서 그림을 추가한다.
\begin{center}
\includegraphics[height=15mm]{basicsOfLatex/figures/basicFigures}
\end{center}
첫번째 파일은 \LaTeX 파일과 동일한 디렉토리에 저장된다. 두번째 파일은 디렉토리 두개 정도 떨어져 있다. 
\vspace{0.5cm} \\
\emph{사용되는 파일과 디렉토리 명칭은 공백이 없어야 한다.}
\vspace{0.5cm} \\
\texttt{[height=1.0in]} 은 선택옵션 인자로 그림 크기를 제어한다.
\end{frame}

\begin{frame} \frametitle{그림 가운데 정렬}
표와 마찬가지로, 그림을 가운데 정렬할 수 있다:
\begin{center}
\includegraphics[height=1.1in]{basicsOfLatex/figures/centerFigure}
\end{center}
두번째 방법은 어떤 점에서 떠다니는 표와 유사한가? (두번째 방법이 떠다니는 그림이다.)
\end{frame}

\begin{frame} \frametitle{캡션과 참조 추가하기}
떠다니는 \texttt{\color{highlight}table} 환경처럼, 떠다니는 그림도 캡션과 참조를 갖을 수 있다.
\begin{center}
\includegraphics[height=0.9in]{basicsOfLatex/figures/figureWithCaptionLabel}
\end{center}
표와 마찬가지로, 참조가 올바르게 동작하도록 \texttt{\color{command}$\backslash$caption} 명령어 다음에 \texttt{\color{command}$\backslash$label} 명령어가 와야 된다.
\vspace{5mm} \\
그림$\sim${\color{command}$\backslash$ref}{\color{braces}$\{${\color{black}figureTemplate}$\}$}을 통해서 참조할 그림 템플릿을 찾는다.
\end{frame}

\subsection[체계적으로 정리]{체계적으로 정리}
\begin{frame} \frametitle{체계적으로 정리}
\LaTeX을 사용할 때 체계적으로 정리하는 비법 몇가지:
\begin{itemize}
\item 디렉토리마다 \LaTeX 문서는 하나.
\item 그림에 사용할 \texttt{\color{command}$\backslash$label} 명칭을 선택할 때, 그림 파일 명칭과 매칭시킨다.
\item 그림 파일을 디렉토리에 체계적으로 정리한다.
\item 만약 코드를 사용해서 그림을 생성한다면, 그림과 동일한 디렉토리에 동일한 명칭(하지만, 확장자는 물론 다르다)으로 코드를 정한다.
\item 기억하라, 파일 혹은 디렉토리 명칭에 공백은 없다.
\end{itemize}
\end{frame}

\frame{ \frametitle{실습}
\texttt{\color{highlight}latexTemp/figures/lower82/} 디렉토리에 이미지를 사용해서, 다음 그림을 생성하시오. 높이 2 센티미터(0.8 인치), 가운데 정렬, 캡션 추가, 참조 추가하세요. 그림을 참조하는 문장도 작성하고 \LaTeX 문서를 두번 컴파일 해서 참조가 정상 동작되도록 한다. \\
\begin{center}
   \includegraphics[height=1.3in]{basicsOfLatex/latexTemp/figures/lower82/lower82}
\end{center}

\texttt{\color{highlight}latex-intro-kr.tex} 파일에 예제를 자유로이 사용하고, \LaTeX  \hspace{0.1cm} 떠다니는 그림 템플릿(Float Figure template)을 사용하세요.
}


\section[추가정보]{추가정보}

\subsection[사례]{사례}
\frame{ \frametitle{다방면의 가능성}
\LaTeX은 다양한 작업을 수행할 강력한 기능이 있다:
\begin{itemize}
\item 화학식
\item 전자회로
\item 파인만 도표(소립자 등의 상호 작용을 표현하는 도표)
\item 음악 악보
\item 십자 낱말 맞추기
\item 스도쿠 퍼즐
\end{itemize}
이러한 다양한 응용프로그램은 일반적으로 다운로드한 기본 \LaTeX 에는 포함되어 있지 않다. 하지만, 비용없이 무료로 다운로드 받을 수 있다. \\
미려한 문서와 그림을 만들 수 있는 능력이 \LaTeX 이 인기를 끄는 이유 중 하나다. 사용자에게 거의 무한대로 문서에 제어권한을 부여한다 (처음에는 다소 제어가 되지 않는다고 느낄 수도 있지만).
}

\subsection[향후 계획?]{향후 계획?}
\frame{ \frametitle{\LaTeX II}
다음 교육과정에서 다음 주제를 다룰 예정이다.
\begin{itemize}
\item \LaTeX 수학
\item BibTeX
\item 사용자 정의 명령어와 환경
\end{itemize}
}

\subsection[책]{책}
\frame{ \frametitle{추천하는 책}
\emph{Guide to LaTeX}, by Helmut Kopka and Patrick W. Daly
}

\subsection[숙제]{숙제}
\frame{ \frametitle{숙제}
만약 이 과정을 학점과정으로 이수하고 있다면, 다음 수업전까지 다음 과제를 제출한다.
\LaTeX 파일 원본 $+$ PDF 파일을 함께 포함해서 제출한다. 종이가 낭비되지 않도록 두 페이지를 종이 한장에 양쪽으로 출력하라. \\[5mm]

\LaTeX 으로 2 페이지 문서를 작성하라(혹은, \LaTeX 이 아닌 또다른 문서를 \LaTeX으로 변환한다). 문서에는 (a) 2개 이상 문단을 포함하고, (b) 적어도 그림 한개를 포함하고, (c) 적어도 표가 하나 있어야 하고, (d) 표와 그림에 {\color{command}$\backslash$label}을 사용해서 텍스트를 참조가 포함되야 한다.
}


\end{document}