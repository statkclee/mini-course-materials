
%READ ME
% LaTeX 명령어 외양을 쉽게 생성하도록 명령어 두개를 생성했다.
%    \lcom{command} 명령어를 사용해서 역슬래쉬와 색깔을 포함하는 명령어를 출력한다. 
%    \larg{argument} 명령어를 사용해서 인자를 녹색 중괄호로 출력한다. 
%    \mathText{math text} 명령어를 사용해서 두 녹색 $ 주위 인자를 출력한다.
% 추가적으로, 명령어를 조합한다:
%    {\color{braces}\$}\texttt{\color{command}$\backslash$alpha}\texttt{\color{braces}\$}
%    대신에
%    \mathText{\lcom{alpha}} 명령어를 사용한다.

\section[수학]{수학}

\subsection[수학]{수학}
\begin{frame} \frametitle{\LaTeX 수학}
\LaTeX 에서 제공하는 수학 환경 몇가지 측면을 다룰 예정이다.
\begin{itemize}
\item 텍스트에 수식 기초
\item 다양한 방정식 환경
\item 수학 기호
\item 수학 표현식
\item 텍스트 강조와 변형
\item 괄호 기호의 자동 크기 조정
\item 수학 방정식에 텍스트
\item 배열과 행렬
\end{itemize}
\end{frame}

\subsection[수식 삽입]{수식 삽입}
\begin{frame} \frametitle{수식을 텍스트에 삽입}

\LaTeX에서 $\alpha$, $\zeta$, $\mu$ ... 같은 그리스 문자를 추가하기 쉽다.
동일한 방식으로 방정식도 쉽게 추가될 수 있다: $y=x^3$, $\sum z^j$, $x_1+\cdots+x_n$.
\begin{center}
\includegraphics[height=0.47in]{math/mathInText}
\end{center}

{\color{braces}\$} 기호는 \LaTeX으로 하여금 언제 수식 모형으로 들어가거나 빠져나올지를 일러준다.
예를 들어, 상기 $\alpha$ 를 생성하려면, \mathText{\lcom{alpha}} 타이핑한다. % {\color{braces}\$}\texttt{\color{command}$\backslash$alpha}\texttt{\color{braces}\$}.
\vspace{5mm} \\
$\beta$을 어떻게 생성할 수 있을까요?
\end{frame}

\begin{frame} \frametitle{방정식 배열}
방정식 일부는 길어서 텍스트와 함께 표기되지 못해 그 차제로 행에 표시되어야 한다. 이런 경우에 \texttt{\color{highlight}eqnarray} 혹은 \texttt{\color{highlight}eqnarray$^*$} 환경을 사용한다:
\begin{center}
\includegraphics[height=0.4in]{math/eqnarrayStar}
\end{center}
\texttt{\color{highlight}eqnarray$^*$}으로 \LaTeX 표현 결과::
\begin{eqnarray*}
\sum_{k=0}^{\infty}0.5^k = \frac{1}{1-0.5} = 2
\end{eqnarray*}
\end{frame}

\begin{frame} \frametitle{방정식 참조}
표와 그림과 마찬가지로, 방정식을 참조할 수 있다. \texttt{\color{highlight}eqnarray} (별표 없음) 을 사용해서 방정식 번호를 붙인다:
\begin{eqnarray}
\sum_{k=0}^{\infty}0.5^k = \frac{1}{1-0.5} = 2
\end{eqnarray}
\texttt{\color{command}$\backslash$label}\texttt{\color{braces}\{}\texttt{powerSeries}\texttt{\color{braces}\}} 라벨을 방정식 배열 내부에 두고 나서, \texttt{\color{command}$\backslash$ref}\texttt{\color{braces}\{}\texttt{powerSeries}\texttt{\color{braces}\}} 을 통해 참조한다.
\begin{center}
\includegraphics[height=0.6in]{math/eqnarray}
\end{center}
\end{frame}

\begin{frame} \frametitle{줄맞춘 방정식}
또다른 환경, \texttt{\color{highlight}align} (그리고 \texttt{\color{highlight}align$^*$}) 은 여러행에 걸친 방정식 줄을 맞추는데 편리하다.
\vspace{-0.3cm}
\begin{center}
\includegraphics[height=14mm]{math/align}
\end{center}
\vspace{-0.3cm}
결과는 다음과 같다:
\vspace{-0.3cm}
\begin{align}
(a+b)^3 &= (a+b) (a^2 + 2ab + b^2) \notag \\
\vspace{-0.1cm}
	      &= a^3 + 3a^2b + 3ab^2 + b^3
\end{align}
\vspace{-0.1cm}
\texttt{\color{command}$\backslash\backslash$} 명령어로 줄바꿈한다. \texttt{\color{command}$\backslash$notag} 명령어로 첫행에 방정식 숫자가 나타나지 못하게 한다. 여기에, \texttt{\color{highlight}amsmath} 팩키지가 필요하다. (질문: 방정식 번호가 있으려면, 코드에 무엇을 포함해야 할까?)
\end{frame}

\begin{frame} \frametitle{다수 줄맞춤}
\texttt{\color{highlight}align} 환경은 다수 정렬을 가능케 한다:
\begin{center}
\includegraphics[height=14mm]{math/alignMany}
\end{center}
출력 결과는 다음과 같다.
\begin{align*}
(a+b)^0 &=1                            & (a+b)^1 &= a+b \\
(a+b)^2 &=a^2 + 2ab + b^2     & (a+b)^3 &= a^3 + 3a^2b + 3ab^2 + b^3
\end{align*}
%The \texttt{\color{command}$\backslash\backslash$} command creates a line break. The command \texttt{\color{command}$\backslash$notag} was used to suppress the equation number of the first line (package \texttt{\color{highlight}amsmath} required).
%\texttt{\color{command}$\backslash$label}\texttt{\color{braces}\{}\texttt{powerSeries}\texttt{\color{braces}\}} can be put inside the equation array and then be referenced via \texttt{\color{command}$\backslash$ref}\texttt{\color{braces}\{}\texttt{powerSeries}\texttt{\color{braces}\}}.
\end{frame}

\subsection[수학과 기호]{수학과 기호}

\begin{frame} \frametitle{수학과 기호}
모든 수학 구문을 배우는 것은 어렵다. \LaTeX 과 Matrix Panels이 도움이 많이 된다:
\begin{center}
\includegraphics[height=1.5in]{math/panels}
\end{center}
Matrix Panel 이 특히 유용한데 배열을 만들려면 작성할 것이 많기 때문이다.
\LaTeX 패널은 신속한 참조용로 편리하게 사용할 수 있다.
\end{frame}

\begin{frame} \frametitle{수학 기호 일부}
\LaTeX에 이용가능한 일부 기호가 다음에 나와 있다.
%\begin{center}
\begin{tabular}{rl p{4mm} rl p{4mm} rl}
$\leftarrow$ & {\color{braces}\${\color{command}$\backslash$leftarrow}\$} &&
$\Leftarrow$ & {\color{braces}\${\color{command}$\backslash$Leftarrow}\$} && 
$\leftrightarrow$ & {\color{braces}\${\color{command}$\backslash$leftrightarrow}\$} \\
$\geq$ & {\color{braces}\${\color{command}$\backslash$geq}\$} &&
$\neq$ & {\color{braces}\${\color{command}$\backslash$neq}\$} &&
$\not\in$ & {\color{braces}\${\color{command}$\backslash$not$\backslash$in}\$} \\
$\partial$ & {\color{braces}\${\color{command}$\backslash$partial}\$} &&
$\oint$ & {\color{braces}\${\color{command}$\backslash$oint}\$} &&
$\nabla$ & {\color{braces}\${\color{command}$\backslash$nabla}\$} \\
$\bigcap$ & {\color{braces}\${\color{command}$\backslash$bigcap}\$} &&
$\bigcup$ & {\color{braces}\${\color{command}$\backslash$bigcup}\$} &&
$\cap$ & {\color{braces}\${\color{command}$\backslash$cap}\$} \\
$\subset$ & {\color{braces}\${\color{command}$\backslash$subset}\$} &&
$\supseteq$ & {\color{braces}\${\color{command}$\backslash$supseteq}\$} &&
$\not\supseteq$ & {\color{braces}\${\color{command}$\backslash$not$\backslash$supseteq}\$} \\
$\bigodot$ & {\color{braces}\${\color{command}$\backslash$bigodot}\$} &&
$\bigotimes$ & {\color{braces}\${\color{command}$\backslash$bigotimes}\$} &&
$\oplus$ & {\color{braces}\${\color{command}$\backslash$oplus}\$} \\
$\clubsuit$ & {\color{braces}\${\color{command}$\backslash$clubsuit}\$} &&
$\perp$ & {\color{braces}\${\color{command}$\backslash$perp}\$} &&
$\vdash$ & {\color{braces}\${\color{command}$\backslash$vdash}\$} \\
\end{tabular}
%\end{center}
수천가지 기호를 검색할 수 있는 PDF 파일은 웹사이트를 참조.
\vspace{-0.3cm}
\begin{itemize}
\item[] {\small\color{highlight}www.ctan.org/tex-archive/info/symbols/comprehensive/symbols-a4.pdf}
\end{itemize}
\vspace{-0.3cm}
({\color{highlight}Window} 메뉴 항목 아래) LaTeX Panel 도 참조한다.
\end{frame}

\begin{frame} \frametitle{문자 변형}
수학 모드에서 텍스트와 기호도 변형할 수 있다.
\begin{center}
\begin{tabular}{rl p{4mm} rl p{4mm} rl}
\hline
\multicolumn{3}{l}{정규} & {변형} &&& {강조} & \\
\hline
{\color{braces}\${\color{black}R}\$} & $R$ &&
{\color{braces}\${\color{command}$\backslash$mathbb}$\{${\color{black}R}$\}$\$} & $\mathbb{R}$ &&
{\color{braces}\${\color{command}$\backslash$tilde}$\{${\color{black}R}$\}$\$} & $\tilde{R}$ \\
{\color{braces}\${\color{black}A}\$} & $A$ &&
{\color{braces}\${\color{command}$\backslash$mathcal}$\{${\color{black}A}$\}$\$} & $\mathcal{A}$ &&
{\color{braces}\${\color{command}$\backslash$widetilde}$\{${\color{black}A}$\}$\$} & $\widetilde{A}$ \\
{\color{braces}\${\color{black}x}\$} & $x$ &&
{\color{braces}\${\color{command}$\backslash$mathbf}$\{${\color{black}x}$\}$\$} & $\mathbf{x}$ &&
{\color{braces}\${\color{command}$\backslash$bar}$\{${\color{black}x}$\}$\$} & $\bar{x}$ \\
{\color{braces}\${\color{black}p}\$} & $p$ &&
{\color{braces}\${\color{command}$\backslash$mathit}$\{${\color{black}p}$\}$\$} & $\mathit{p}$ &&
{\color{braces}\${\color{command}$\backslash$hat}$\{${\color{black}p}$\}$\$} & $\hat{p}$ \\
{\color{braces}\${\color{black}X}\$} & $X$ &&
{\color{braces}\${\color{command}$\backslash$mathrm}$\{${\color{black}X}$\}$\$} & $\mathrm{X}$ &&
{\color{braces}\${\color{command}$\backslash$widehat}$\{${\color{black}X}$\}$\$} & $\widehat{X}$ \\
\hline
\end{tabular}
\end{center}
두가지 다른 강조:  {\color{braces}\${\color{command}$\backslash$dot}$\{${\color{black}x}$\}$\$} 와 {\color{braces}\${\color{command}$\backslash$ddot}$\{${\color{black}x}$\}$\$} 으로 $\dot{x}$ 와 $\ddot{x}$.
\end{frame}

\begin{frame} \frametitle{첨자와 지수}
윗첨자(예를 들어, $x_1$)와 아래첨자(예를 들어, $3^2$)도 생성할 수 있다:
\begin{center}
\includegraphics[height=8mm]{math/subSuperscript}
\end{center}
첨자가 단일 문자일 때, 괄호를 생략할 수 있다. 즉, 상기 텍스트에 대해 다음도 동일하게 받아들일 수 있다:
\begin{center}
\includegraphics[height=8mm]{math/subSuperscriptNoBraces}
\end{center}
만약 윗/아래 첨자 문자가 하나 이상이라면, 괄호를 사용해서 문제를 피한다: {\color{braces}\${\color{black}2\textbf{\_}10}\$} 은 $2_10$ 을 만들어 낸다. 위첨자와 아래첨자를 동시에 사용할 수 있다: $x_{ij}^2$.
\end{frame}

\begin{frame} \frametitle{분수와 근(root)}
$\frac{2+3}{4+5} = \frac{5}{9}$ 처럼 분수 혹은 $\sqrt{81}=9$ 와 $\sqrt[4]{81} = 3$ 처럼 근(root)도 쉽게 생성할 수 있다.
\begin{center}
\includegraphics[height=7mm]{math/fracRoots}
\end{center}
물론 분수와 근도 조합할 수 있다:
$\frac{\sqrt{4} + 3}{\sqrt{16} + 5} = \frac{5}{9}$.
\begin{center}
\includegraphics[height=7mm]{math/fracRootsCombined}
\end{center}
\end{frame}

\begin{frame} \frametitle{합과 적분}
합과 적분도 생성할 수 있다:
\vspace{-0.3cm}
\begin{center}
\includegraphics[height=14mm]{math/sumIntegral}
\end{center}
\vspace{-0.3cm}
결과는 다음과 같다
\vspace{-0.3cm}
\begin{align*}
\sum_{i=0}^{\infty} p^i &= \frac{1}{1-p}   &  \int_{1}^{2}3x^2dx          &= 7 \\
\sum\nolimits_{i=0}^{\infty} 0.5^i &= 2    &  \int\limits_{1}^{1}3x^2dx &= 0
\end{align*}

\texttt{\color{command}$\backslash$nolimits} 와 \texttt{\color{command}$\backslash$limits} 명령어를 사용해서 \LaTeX 에 디폴트 기본설정된 극한(limit) 표시를 덮어쓸 수 있다.
\end{frame}

\begin{frame} \frametitle{실습}
\texttt{\color{highlight}eqnarray$^*$} 환경을 사용해서 다음 결과를 생성하시오:
\begin{center}
   \includegraphics[height=0.6in]{math/tryIt3}
\end{center}
예제 일부를 \texttt{\color{highlight}latex-intro-kr.tex} 파일에서 활용가능하다.
\end{frame}

\subsection[최종 항목]{최종항목}
\begin{frame} \frametitle{괄호 크기}
괄호 크기와 관련한 작은 문제가 왼쪽 식에 나와 있고, 오른편에 문제를 고친 식이 나와 있다.
\vspace{-0.3cm}
\begin{align*}
   (\frac{2+3}{4+5}) &&& \left(\frac{2+3}{4+5}\right)
\end{align*}
상기 표현식에 사용된 코드는 다음과 같다.
\begin{center}
\includegraphics[height=9mm]{math/fixingParentheses}
\end{center}
\vspace{-0.3cm}
일반적으로 자동으로 크기 조절되는 괄호를 생성하려면, {\color{command}$\backslash$left\color{black}(}, {\color{command}$\backslash$left\color{black}[}, {\color{command}$\backslash$left\color{black}$|$},  {\color{command}$\backslash$left$\backslash\{$} 와 함께 상응하는 오른쪽 괄호를 함께 사용한다. 상기 명령어는 방정식 환경 내부에 \emph{있어야만} 하고, 왼쪽과 오른쪽 괄호는 항상 짝이 맞아야 된다.
\end{frame}

\begin{frame} \frametitle{행렬}
행렬도 \LaTeX에서 만들 수 있다:
\begin{eqnarray*}
\left(\begin{array}{ccc} 4 & 1 & 19 \\ 3 & 8 & 8\end{array}\right)
\end{eqnarray*}
상기 행렬을 만드는 코드는 다음과 같다:
\begin{center}
\includegraphics[height=0.6in]{math/array}
\end{center}
\texttt{\color{highlight}array}에 대한 구문은 \texttt{\color{highlight}tabular} (표)에 사용하는 구문과 같다.
\end{frame}

\begin{frame} \frametitle{공백과 쌓기(stacking)}
{\color{command}$\backslash$quad} 을 사용해서 방정식에 공백을 추가할 수 있고, {\color{command}$\backslash$stackrel} 을 사용해서 표현식을 쌓을 수도 있다:

\vspace{1mm}

\begin{itemize}
\item[] {\color{command}$\backslash$begin}{\color{braces}$\{${\color{black}eqnarray$^*$}$\}$}
\item[] \hspace{2mm} E(X+Y)
	{\color{command}$\backslash$stackrel\color{braces}$\{${\color{black}indep.}$\}\{${\color{black}=}$\}$}
	E(X) + E(Y)
\item[] \hspace{5mm} {\color{command}$\backslash$quad$\backslash$quad}
\item[] \hspace{5mm} Var(X+Y)
	{\color{command}$\backslash$stackrel\color{braces}$\{${\color{black}indep.}$\}\{${\color{black}=}$\}$}
	Var(X) + Var(Y)
\item[] {\color{command}$\backslash$end}{\color{braces}$\{${\color{black}eqnarray$^*$}$\}$}
\end{itemize}

\vspace{1mm}

상기 명령어는 다음을 만들어 낸다:
\begin{eqnarray*}
E(X+Y) \stackrel{indep.}{=} E(X) + E(Y) \quad\quad Var(X+Y) \stackrel{indep.}{=} Var(X) + Var(Y)
\end{eqnarray*}
\end{frame}

