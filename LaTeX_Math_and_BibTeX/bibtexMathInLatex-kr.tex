\documentclass[slidestop,compress,mathserif]{beamer}
\usetheme{Frankfurt}
\usecolortheme{seagull}

% Load packages
\usepackage{alltt}
\usepackage{verbatim}
\usepackage{geometry}                % See geometry.pdf to learn the layout options. There are lots.
\usepackage{graphicx}
\usepackage{amssymb}
\usepackage{amsmath}
\usepackage{epstopdf}
\usepackage{verbatim}
%\usepackage{musixtex}
%\usepackage{xymtexps}
\usepackage{feynmp}

%\usepackage{pgfpages}
%\pgfpagesuselayout{4 on 1}[letterpaper,landscape,border shrink=5mm]
\usepackage[hangul]{kotex}
\usepackage[T1]{fontenc}
\usepackage[utf8]{inputenc}


\DeclareGraphicsRule{.tif}{png}{.png}{`convert #1 `dirname #1`/`basename #1 .tif`.png}


% Title 
% Note: [short title]{long title}, [short author(s) name]{long author(s) name}
\title{\LaTeX 소개 2부 : 수학, BibTeX, 사용자 정의}
\subtitle{공개 통계학 개론 (OpenIntro Stat.) 저작 학습용}

\author{David Diez \inst{1} \and 이광춘\scriptsize{(번역)} \inst{2}} 
\institute{\inst{1} OpenIntro \href{http://www.openintro.org}{openintro.org} \and \inst{2} xwMOOC \href{http://www.xwmooc.net}{xwmooc.net}}
\date{}

\begin{document}
\definecolor{highlight}{rgb}{.7,.1,.1}
\definecolor{command}{rgb}{.1,.1,.9}
\definecolor{comment}{rgb}{1,0,0}
\definecolor{braces}{rgb}{0,0.5,0}
\newenvironment{act}[1]{{\color{command}#1}}{}
\newcommand{\lcom}[1]{{\color{command}$\backslash$#1}}
\newcommand{\larg}[1]{{\color{braces}$\{${\color{black}#1}$\}$}}
\newcommand{\mathText}[1]{{\color{braces}\${\color{black}#1}\$}}


\frame{ \titlepage }

\begin{frame}
  \frametitle{목차}
  \begin{itemize}
  \item \LaTeX 수학
  \item BibTeX: \LaTeX 참고문헌
  \item 사용자 정의 명령어와 환경 만들기
  \item 기타 비법
  \end{itemize}
\end{frame}

\begin{frame}  \frametitle{Guide to LaTeX}
\textit{Guide to LaTeX} 책에는 \LaTeX에 대한 멋진 안내가 나와 있고, 이번 학습에서 이 책에서 나온 예제 일부를 충실히 따라간다:
\begin{itemize}
\item[7] 수학
\item[11,12] BibTeX
\item[10] 사용자 정의 명령어와 환경
\end{itemize}

\LaTeX에 관한 학습교재를 찾는다면, \textit{Guide to LaTeX} 책은 훌륭한 대안이 될 수 있다.
\end{frame}

\part{}

%\include{math/math-kr}
%\section[BibTeX]{BibTeX}

\subsection[BibTeX에 오신걸 환영합니다.]{BibTeX에 오신걸 환영합니다.}

\begin{frame}  \frametitle{왜 BibTeX를 사용할까요?}
참고문헌을 수작업으로 생성하는 대신에 BibTeX를 사용할 매우 좋은 이유가 상당수 있다.
\begin{itemize}
\item 참고문헌 자동 생성.
\item 쉬운 참고문헌 스타일 변경.
\item 텍스트에서 참조하면서 빼먹은 참고문헌 식별.
\end{itemize}
\end{frame}

\begin{frame}  \frametitle{BibTeX은 어떻게 동작하나}
참고문헌을 만드는데 \LaTeX 과 BibTeX 은 세 단계를 밟는다.
\begin{itemize}
\item 인용문을 갖는 문서를 조판할 때 (예를 들어, {\color{command}$\backslash$cite{\color{braces}$\{$}{\color{black}zotova}{\color{braces}$\}$}}), \LaTeX은 인용문 각각에 대해 메모한다.
\item BibTeX 은 작성된 목록을 갖고 출판 데이터베이스에서 각각의 참조를 찾는다.
\item 그리고 나서, \LaTeX에 지시해서 참조하려고 찾은 모든 출판물에 대한 참고문헌을 작성하게 한다.
\end{itemize}
가장 시간이 많이 소요되는 부분이 최초 데이터베이스를 빌드할 때다. 그 다음부터 동일한 데이터베이스를 반복해서 참조하고, BibTeX은 식은 죽 먹듯 쉽게된다.
\end{frame}

\begin{frame}  \frametitle{BibTeX 재료}
\begin{itemize}
\item 데이터베이스 생성한다.
\item 참조문헌 인용한다.
\item BibTeX으로 조판한다.
\item 스타일 파일을 빌드한다.
\end{itemize}
\end{frame}

\subsection[데이터베이스를 빌드한다]{데이터베이스를 빌드한다}

\begin{frame}  \frametitle{참조 표본 입력}
다음과 유사하게 인용하려는 각 항목마다 참조문헌을 생성한다.

\vspace{7mm}

@article$\{\overbrace{\text{zotova}}^{label}$, \\
\hspace{3mm}	Author = $\{$Elena Zotova and Charles D Woody and Ehud Gruen$\}$, \\
\hspace{3mm}	Journal = $\{$Brain Research$\}$, \\
\hspace{3mm}	Pages = $\{$66-78$\}$, \\
\hspace{3mm}	Title = $\{$Multiple representations ... [etc etc].$\}$, \\
\hspace{3mm}	Volume = $\{$868$\}$, \\
\hspace{3mm}	Year = $\{$2000$\}\}$ \\
\end{frame}

\begin{frame}  \frametitle{참고문헌 구성}
각 참고문헌은 출판유형(예를 들어, 책, 기사) 정보가 필요하고, 많은 필드를 포함한다. 예를 들어, 기사(article)입력에 필요한 \textbf{필수}(required)필드와 \textit{선택옵션}(optional) 필드가 다음에 나와 있다.
\begin{center}
\begin{tabular}{lllrr}
\multicolumn{5}{l}{\textbf{라벨}: 참고문서 라벨.} \\
\textbf{저자(Author)} & \textbf{저널(Journal)} & \textbf{제목(Title)} & \hspace{5mm} & \\
\textbf{년도(Year)} & \textit{볼륨(Volume)} & \textit{숫자(Number)} \\
\textit{쪽(Pages)} & \textit{월(Month)} & \textit{메모(Note)}
\end{tabular}
\end{center}

이용가능한 출판 유형 형식목록과 더불어 각 유형에 대해 어느 항목이 필요하고, 어느 항목이 선택옵션인지 확인한다.
\begin{itemize}
\item[]\color{highlight}http://www.image.ufl.edu/help/latex/entry\_bibtex.shtml
\end{itemize}
\end{frame}

\begin{frame}  \frametitle{수작업에 대한 대안}
만약 데이터베이스를 이러한 방식으로 만들고 싶지 않는다면, 다음 서지관리 소프트웨어를 시도해 본다.
\begin{itemize}
\item BibDesk: 맥
\item JabRef: 맥, 윈도우, 리눅스
\end{itemize}
BibDesk과 JabRef 서지관리 프로그램 모두 무료이며 온라인에서 이용가능하다. 다른 프로그램도 존재하고 있지만, 맥을 사용하는 저자는 개인적으로 BibDesk만 사용해봤다.
\end{frame}

\begin{frame}  \frametitle{본인 스스로 관리를 한다면...}
만약 서지관리 프로그램 관리자를 사용하지 않는다면, 알아야 되는 것이 몇가지 있다:
\begin{itemize}
\item 항상 라벨을 포함하는데, 라벨은 \LaTeX이 입력항목을 식별하는 방법이다.
\item 입력항목 (출판) 유형과 필드명은 대소문자 구분하지 않는다.
\item 필드 각각(예를 들어, 저자명)에 대한 텍스트를 중괄호 내부에 감싼다.
\item 목록에 등재되지 않는 추가 필드를 추가할 수 있지만, 이러한 필드는 무시된다 (예를 들어, 만약 참고문헌에 초록(Abstract) 필드를 추가하지만, BibTeX는 이를 무시한다).
\end{itemize}
\end{frame}

\begin{frame}  \frametitle{특별한 경우}
모호하지 않은 형태로 저자명을 부여하기가 때로는 까다롭다.
\begin{itemize}
\item 항상 이름을 $\{$이름(Given Names) 성(Surnames)$\}$ 혹은 $\{$성(Surname), 이름(Given Names)$\}$ 으로 타이핑한다.
\item 괄호안에 감싼 어떤 것이나 단일 항목으로 취급된다 (예를 들어, 저자 = {\color{braces}$\{${\color{black}Maria} $\{${\color{black}San Martino}$\}\}$}).
\item 만약 저자가 한명 이상이면, 저자명 각각을 단어 \textit{and}로 분리. \textit{and}가 누군가 이름의 일부라면, 전체 이름을 괄호로 감싼다.
\item 강세를 추가할 수도 있다 (즉, \texttt{G$\{\backslash$"o$\}$del} 명령어로 G$\ddot{\text{o}}$del).
\end{itemize}
다른 많은 뉴앙스가 존재한다. 만약 특별한 이름과 마주치면, 데이터베이스에 넣을 최선의 방법을 온라인 검색한다.
\end{frame}

\begin{frame}  \frametitle{저널 명칭 축약}
종종 입력항목에 전체 저널 명칭을 넣고 싶지 않을 때가 있다. 저널 명칭을 축약하려면, \textit{string} 입력 유형을 사용한다:
\begin{itemize}
\item[] @string$\{$JSS = $\{$Journal of Statistical Software$\}\}$
\end{itemize}
축약된 저널 명칭이 사용될 데이터베이스에 이러한 문자열 입력항목이 정의되어야만 된다.
\end{frame}

\subsection{참고문헌 참조}

\begin{frame}  \frametitle{참고문헌 참조}
사용될 수 있는 명령어가 네가지가 있따.
\begin{itemize}
\item {\color{command}$\backslash$cite\color{braces}$\{${\color{black}라벨이름}$\}$} [참조번호], 예를 들어 [1].
\item {\color{command}$\backslash$citet\color{braces}$\{${\color{black}라벨이름}$\}$} 성 (년도), 예를 들어 Zotova et al. (2000).
\item {\color{command}$\backslash$citep\color{braces}$\{${\color{black}라벨이름}$\}$} (성, 년도), 예를 들어, (Zotova et al., 2000).
\item {\color{command}$\backslash$nocite\color{braces}$\{${\color{black}라벨이름}$\}$} 인용되지는 않으나 참고문헌에는 나타남.
\end{itemize}
첫번째와 마지막은 \texttt{\color{highlight}uclathes} 클래스에 동작한다. 가운데 두개는 \texttt{\color{highlight}natbib} 팩키지에 사용된다 (학위논문이 아닌 경우 강력히 추천).
\end{frame}

\begin{frame}  \frametitle{본인 문서에 다른 명령어들}
참고문헌이 추가되는 장소에 다음 코드 두 줄을 삽입해야만 된다:
\begin{itemize}
\item[] {\color{command}$\backslash$bibliographystyle\color{braces}$\{${\color{black}선택한-스타일}$\}$}
\item[] {\color{command}$\backslash$bibliography\color{braces}$\{${\color{black}데이터베이스-명칭}$\}$}
\end{itemize}
스타일 명령어는 좀더 위로 이동할 수 있다 (문제가 되지는 않는다). 만약 \texttt{\color{highlight}natbib} 팩키지를 사용한다면, 다른 팩키지와 함께 추가한다:
\begin{itemize}
\item[] {\color{command}$\backslash$usepackage\color{braces}$\{${\color{black}natbib}$\}$}
\end{itemize}

\end{frame}

\subsection{조판(Typesetting)}

\begin{frame}  \frametitle{참고문헌 만들기}
만약 참고문헌 데이터베이스를 만들고, 인용을 만들고, 텍스트에 참고문헌 참조 명령어를 삽입했다면, 참고문헌을 생성할 준비를 마쳤다. TeXShop에, 마무리하는 간단한 몇단계 절차가 있다:
\begin{itemize}
\item \LaTeX 문서를 보통때처럼 조판한다.
\item \LaTeX 에서 BibTeX으로 조판 선택옵션을 변경한다:

\vspace{1mm}

\includegraphics[height=15mm]{bibtex/bibtexTypeset}\hspace{4mm}{}

\item BibTeX 으로 다시 조판한다.
\item 다시 조판 선택옵션을 \LaTeX 으로 되돌려서 \textbf{두번} 이상 컴파일한다.
\end{itemize}
\end{frame}

\subsection{스타일 파일 빌드하기 (쉬운 방법)}

\begin{frame}  \frametitle{스타일 파일 빌드하기}
BibTeX을 사용하는 큰 장점 중 하나는 참고문헌 스타일과 텍스트 내부 인용을 신속히 변경할 수 있는 역량이다.
이를 위해서 {\color{highlight}custom-bib} 프로그램을 사용한다. 다음 웹사이트에서 다운로드한다.
\begin{itemize}\small 
\item[] \color{highlight} http://www.ctan.org/tex-archive/help/Catalogue/entries/custom-bib.html
\end{itemize}
custom-bib 팩키지는 {\color{highlight}latexTemp} zip 앞축파일에 포함되어 있다.
\end{frame}

\begin{frame}  \frametitle{스타일 파일 빌드하기}
\texttt{latexTemp} $>$ \texttt{custom-bib} 디렉토리를 열고, \texttt{\color{highlight}makebst.tex} 파일을 연다. 프로그램을 실행시키기 위해서, 
\begin{itemize}
\vspace{-0.2cm} \item[(1)] 파일을 열어서 조판한다.
\vspace{-0.2cm} \item[(2)] 추가 디렉토를 얻어오는 첫번째 질문에 \texttt{YES}를 타이핑한다.
\vspace{-0.2cm} \item[(3)] 적절한 파일 명칭을 선택한다 (확장자를 추가할 필요는 없다).
\vspace{-0.2cm} \item[(4)] 각 스타일 질문에 대답한다.
\vspace{-0.2cm} \item[(5)] 마지막 질문에, \textit{Finished!! ...  Shall I now run this batch job? (NO)}, \texttt{YES} 를 타이핑한다.
\end{itemize}
(3) 단계에서 명명한 \texttt{.bst} 확장자를 갖는 파일을 찾아 복사한다. 해당 스타일을 적용할 참고문헌을 생성할 파일을 갖는 폴더에 두거나 혹은 참고문헌 폴더에 둔다 (\LaTeX 문서에 스타일 파일로 이것을 참조한다).
\end{frame}

\subsection{실습}

\begin{frame}  \frametitle{실습}
\texttt{\color{highlight}latex-intro-kr.tex} 파일을 열고 마지막 절로 간다. {\color{command}$\backslash$citet}{\color{braces}$\{${\color{black}victor}$\}$}의 참고문헌을 추가한다. 또한 {\color{command}$\backslash$citep}{\color{braces}$\{${\color{black}victor}$\}$} 로 참조대상을 추가하고 조판한다 (4단계 모든 단계). 참고문헌 사이에 차이는 무엇인가요? 논문에서 각각을 어떻게 사용할까요?
\end{frame}




\section[사용자 정의]{사용자 정의}

\subsection[다룰 내용]{다룰 내용}

\begin{frame}  \frametitle{사용자 정의 명령어 재료}
\begin{itemize}
\item 계수기(Counters)
\item 명령어 생성하기
\item 환경 생성하기
\end{itemize}
\end{frame}

\subsection[계수기(Counters)]{계수기(Counters)}

\begin{frame}  \frametitle{기존 계수기}
\LaTeX 은 계수기(변수)를 사용해서 적절하게 번호를 매긴다. 
\begin{itemize}
\item \LaTeX에서 사용되는 계수가 다음에 나와있다: \texttt{\color{highlight}part}, \texttt{\color{highlight}chapter}, \texttt{\color{highlight}section}, \texttt{\color{highlight}subsection}, \texttt{\color{highlight}subsubsection}, \texttt{\color{highlight}page}, \texttt{\color{highlight}footnote}, \texttt{\color{highlight}equation}, \texttt{\color{highlight}figure}, \texttt{\color{highlight}table}. 이러한 계수기는 상응하는 명령어에 대응된다.
\item 각 수준별 열거할 때 다른 계수기가 사용된다: \texttt{\color{highlight}enumi}, \texttt{\color{highlight}enumii}, \texttt{\color{highlight}enumiii}, \texttt{\color{highlight}enumiv}.
\item 다른 \LaTeX \hspace{0.15cm}계수기 몇가지: \texttt{\color{highlight}paragraph}, \texttt{\color{highlight}subparagraph}, \texttt{\color{highlight}mpfootnote}.
\end{itemize}
\end{frame}

\begin{frame}  \frametitle{신규 계수기 생성하기}
본인 목적에 맞춰서 자신만의 계수기를 생성할 수도 있다.
아마도 번호 매기고 싶은 예제가 있을 수 있다.
\begin{itemize}
\item[] {\color{command}$\backslash$newcounter\color{braces}$\{${\color{black}counterName}$\}$\color{black}[inCounter]}
\end{itemize}
상기 명령어는 \texttt{counterName} 으로 불리는 새로운 계수기를 생성한다. \texttt{inCounter} 인자와 $+$ 꺾쇠 괄호는 선택옵션이다. \texttt{inCounter}가 증가할 때마다 \texttt{inCounter} 를 사용해서 \texttt{counterName} 을 재설정한다. (예를 들어,  \texttt{subsection} 은 ``inCounter'' \texttt{section} 을 갖고 있다).
\end{frame}

\begin{frame}  \frametitle{변형하기}
기존 혹은 신규 계수기를 변형할 수 있다.
\begin{itemize}
\item[] {\color{command}$\backslash$setcounter\color{braces}$\{${\color{black}counter}$\}\{${\color{black}n}$\}$}
\item[] {\color{command}$\backslash$addtocounter\color{braces}$\{${\color{black}counter}$\}\{${\color{black}n}$\}$}
\item[] {\color{command}$\backslash$stepcounter\color{braces}$\{${\color{black}counter}$\}$}
\item[] {\color{command}$\backslash$refstepcounter\color{braces}$\{${\color{black}counter}$\}$}
\end{itemize}
만약 {\color{command}$\backslash$label} 을 붙여놓는다면, {\color{command}$\backslash$refstepcounter} 명령어를 사용해서 계수기 값을 참조한다.
\end{frame}

\begin{frame}  \frametitle{출력하기}
계수기를 생성하고, 변형하고, 참조할 수 있다. 하지만, 문서에서 계수기를 출력할 필요도 있다. 계수기를 다음 명령어 중 하나를 호출해서 출력한다:
\newcounter{temp}
\setcounter{temp}{4}
\begin{itemize}
\item[] {\color{command}$\backslash$arabic}{\color{braces}$\{${\color{black}chapter}$\}$} (\arabic{temp}, 아라비아 숫자)
\item[] {\color{command}$\backslash$Roman} (\Roman{temp}, 대문자 로마 숫자)
\item[] {\color{command}$\backslash$roman} (\roman{temp}, 소문자 로마 숫자)
\item[] {\color{command}$\backslash$Alph} (\Alph{temp}, 영문 대문자)
\item[] {\color{command}$\backslash$alph} (\alph{temp}, 영문 소문자)
\item[] {\color{command}$\backslash$fnsymbol} (\fnsymbol{temp}, 각주 기호)
\end{itemize}
사용자 정의 명령어와 환경에 계수기를 두고 사용한다.
\end{frame}

\subsection[명령어]{명령어}

\begin{frame}  \frametitle{간단한 명령어}
\newcommand{\xvec}{x_1, \ldots, x_n}
$x_1, ..., x_n$ 같은 흔한 문장은 다음 new 명령어로 축약할 수 있다.
\begin{itemize}
\item[] {\color{command}$\backslash$newcommand\color{braces}$\{${\color{command}$\backslash$xvec}$\}\{${\color{black}x\_1,{\color{command}$\backslash$dots},x\_n}$\}$}
\end{itemize}
상기 명령어를 삽입하고 나서 (나중에 문서에) {\color{braces}\$\color{command}$\backslash$xvec\color{braces}\$} 을 타이핑하면, $\xvec$ 을 얻게 된다. 만약 달러 부호를 생략하면 문제가 된다. 이것을 해결하는데 추가 명령어를 사용한다.
\begin{itemize}
\item[] {\color{command}$\backslash$newcommand\color{braces}$\{${\color{command}$\backslash$xvec}$\}\{${\color{command}$\backslash$ensuremath{\color{braces}$\{$}{\color{black}x\_1,{\color{command}$\backslash$dots},x\_n}}$\}$ $\}$}
\end{itemize}
두번째 정의 말미에 추가 공백을 두어서, 공백 문제를 방지하게 했다. 좀더 멋진 해결책은 {\color{highlight}xspace} 팩키지를 사용하는 것이다 (\textit{Guide to LaTeX}, 186쪽을 참조한다).
\end{frame}

\newcommand{\subvec}[2]{\ensuremath{#1_{1}, \ldots, #1_{#2}} }
\begin{frame}  \frametitle{인자를 갖는 명령어}
만약 작성한 명령어를 일반화하려고 하면, 인자 두개를 추가한다.
\begin{itemize}
\item[] \hspace{-1mm}{\color{command}$\backslash$newcommand\color{braces}$\{${\color{command}$\backslash$subvec}$\}${\color{black}[2]}$\{${\color{command}$\backslash$ensuremath{\color{braces}$\{$}{\color{black}\#1\_1,{\color{command}$\backslash$dots},\#1\_{\color{braces}$\{$}\#2{\color{braces}$\}$}}}$\}$ $\}$}
\end{itemize}

{\color{command}$\backslash$subvec\color{braces}$\{${\color{black}y}$\}\{${\color{black}m}$\}$}로부터 \subvec{y}{m} 을 생성할 수 있다.

\vspace{7mm}

$n^{th}$ 인자에 대해서 \#n 을 통해 추가 인자를 생성하고 참조할 수 있다. 디폴트 선택옵션 인자도 활용할 수 있다 (\textit{Guide to LaTeX}, 188쪽을 참조한다).
\end{frame}

\begin{frame}  \frametitle{일반화}
신규 명령어에 대한 일반적인 프레임워크는 다음과 같다.
\vspace{0.5mm} \\
\begin{itemize}
\item[] {\color{command}$\backslash$newcommand\color{braces}$\{${\color{command}$\backslash$commandName}$\}${\color{black}[n]}$\{${\color{highlight}the commands}$\}$}
\end{itemize}
\vspace{0.5mm}
여기서
\vspace{0.5mm} \\
\begin{itemize}
\item \texttt{commandName}은 명령어 명칭이 된다.
\item \texttt{n}은 인자 갯수가 된다.
\item 인자를 {\color{highlight}the commands} 에 \#1, \#2, \dots, \#n 으로 참조한다.
\end{itemize}
\vspace{0.5mm}
이미 존재하는 명령어를 다시 정의하려면, 상기와 동일한 형식으로 {\color{command}$\backslash$renewcommand} 명령어를 사용한다.
\end{frame}

\subsection[환경]{환경}

\begin{frame}  \frametitle{견본 환경}
환경은 begin과 end 태그를 사용한다 (예를 들면, \texttt{itemize}). {\color{command}$\backslash$begin} 과 {\color{command}$\backslash$end} 태그에 발생하는 것만 정의할 필요가 있다. 예를 들어, 
\vspace{-2mm}
\begin{itemize}
\item[] {\color{command}$\backslash$newenvironment\color{braces}$\{${\color{black}예제}$\}$}
\item[] \hspace{2mm}{\color{braces}$\{${\color{command}$\backslash$small$\backslash$textbf}$\{${\color{black}예제.}$\}$ {\color{command}$\backslash$hspace}$\{${\color{black}2mm}$\}\}$ {\color{red}\% 시작하는 것}}
\item[] \hspace{2mm}{\color{braces}$\{${\color{command}$\backslash\backslash$}$\}$ {\color{red}\% 끝나는 것}}
\end{itemize}
견본 환경 호출:
\vspace{-2mm}
\begin{itemize}
\item[] {\color{command}$\backslash$begin\color{braces}$\{${\color{black}예제}$\}$}
\item[] 나머지 연산의 덧셈은 신비롭게 동작한다: {\color{braces}\$}2+2=1{\color{braces}\$} (mod 3).
\item[] {\color{command}$\backslash$end\color{braces}$\{${\color{black}예제}$\}$}
\end{itemize}
\vspace{-2mm}
결과:
\vspace{-1mm}\\
\small\textbf{예제.}\hspace{1.5mm} 나머지 연산의 덧셈은 신비롭게 동작한다: $2+2 = 1$ (mod 3).
\end{frame}

\begin{frame}  \frametitle{General environment}
Generally environments take the form
\vspace{0.5mm} \\
\begin{itemize}
\item[] {\color{command}$\backslash$newenvironment\color{braces}$\{${\color{black}environmentName}$\}\{${\color{highlight}begin stuff}$\}\{${\color{highlight}end stuff}$\}$}
\end{itemize}
\vspace{0.5mm}
We can also declare that there will be \texttt{n} arguments.
\vspace{0.5mm} \\
\begin{itemize}
\item[] {\color{command}$\backslash$newenvironment\color{braces}$\{${\color{black}environmentName}$\}${\color{black}[n]}$\{${\color{highlight}begin stuff}$\}\{${\color{highlight}end stuff}$\}$}
\end{itemize}
\vspace{0.5mm}
As before, we refer to the arguments as \#1, \dots, \#n in the {\color{highlight}begin stuff} and {\color{highlight}end stuff}.
\vspace{7mm} \\
To redefine an environment that already exists, use {\color{command}$\backslash$renewenvironment} with the same format as above.
\end{frame}

\begin{frame}  \frametitle{Environment + counter}
\begin{itemize}
\item[] {\color{command}$\backslash$newcounter\color{braces}$\{${\color{black}example}$\}$}
\item[] {\color{command}$\backslash$setcounter\color{braces}$\{${\color{black}example}$\}$$\{${\color{black}0}$\}$}
\item[] {\color{command}$\backslash$newenvironment\color{braces}$\{${\color{black}example}$\}$}
\item[] \hspace{4mm}{\color{braces}$\{${\color{command}$\backslash$refstepcounter\color{braces}$\{${\color{black}example}$\}$}\color{command}$\backslash$small}
\item[] \hspace{8mm}{\color{braces}{\color{command}$\backslash$textbf}$\{${\color{black}Example} {\color{command}$\backslash$arabic\color{braces}$\{${\color{black}example}$\}$}{\color{black}.}$\}${\color{command}$\backslash$hspace}$\{${\color{black}2mm}$\}\}$}
\item[] \hspace{4mm}{\color{braces}$\{${\color{command}$\backslash\backslash$}$\}$}
\end{itemize}
\end{frame}



\section[Wrap-up]{Wrap-up}
\subsection[Wrap-up]{Wrap-up}

\begin{frame}  \frametitle{Organizer and time saver}
The {\color{command}$\backslash$include} command is useful for long documents:
\vspace{1mm} \\
\begin{itemize}
\item[] {\color{command}$\backslash$include}{\color{braces}$\{${\color{black}otherDocName}$\}$}
\end{itemize}
\vspace{1mm}
For instance, this presentation actually calls three separate documents: one for each big section. Thus I would not take time Typesetting parts of the document I was not working on while keeping organized:
\vspace{1mm} \\
\begin{itemize}
\item[] \lcom{include}\larg{math/math} {\color{red}\% ``math'' document in the ``math'' folder}
\item[] {\color{red}\%$\backslash$include$\{${bibtex/bibtex}$\}$}
\item[] {\color{red}\%$\backslash$include$\{${comenv/comenv}$\}$}
\end{itemize}
\end{frame}

\begin{frame}  \frametitle{Wrap-up}
After this class, you should have a general idea of
\vspace{1mm} \\
\begin{itemize}
\item using the math modes in LaTeX,
\item creating bibliographies using BibTeX, and
\item creating your own commands and environments.
\end{itemize}
\vspace{1mm}
Any questions?
\end{frame}





\end{document}