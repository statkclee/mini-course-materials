\section[사용자 정의]{사용자 정의}

\subsection[다룰 내용]{다룰 내용}

\begin{frame}  \frametitle{사용자 정의 명령어 재료}
\begin{itemize}
\item 계수기(Counters)
\item 명령어 생성하기
\item 환경 생성하기
\end{itemize}
\end{frame}

\subsection[계수기(Counters)]{계수기(Counters)}

\begin{frame}  \frametitle{기존 계수기}
\LaTeX 은 계수기(변수)를 사용해서 적절하게 번호를 매긴다. 
\begin{itemize}
\item \LaTeX에서 사용되는 계수가 다음에 나와있다: \texttt{\color{highlight}part}, \texttt{\color{highlight}chapter}, \texttt{\color{highlight}section}, \texttt{\color{highlight}subsection}, \texttt{\color{highlight}subsubsection}, \texttt{\color{highlight}page}, \texttt{\color{highlight}footnote}, \texttt{\color{highlight}equation}, \texttt{\color{highlight}figure}, \texttt{\color{highlight}table}. 이러한 계수기는 상응하는 명령어에 대응된다.
\item 각 수준별 열거할 때 다른 계수기가 사용된다: \texttt{\color{highlight}enumi}, \texttt{\color{highlight}enumii}, \texttt{\color{highlight}enumiii}, \texttt{\color{highlight}enumiv}.
\item 다른 \LaTeX \hspace{0.15cm}계수기 몇가지: \texttt{\color{highlight}paragraph}, \texttt{\color{highlight}subparagraph}, \texttt{\color{highlight}mpfootnote}.
\end{itemize}
\end{frame}

\begin{frame}  \frametitle{신규 계수기 생성하기}
본인 목적에 맞춰서 자신만의 계수기를 생성할 수도 있다.
아마도 번호 매기고 싶은 예제가 있을 수 있다.
\begin{itemize}
\item[] {\color{command}$\backslash$newcounter\color{braces}$\{${\color{black}counterName}$\}$\color{black}[inCounter]}
\end{itemize}
상기 명령어는 \texttt{counterName} 으로 불리는 새로운 계수기를 생성한다. \texttt{inCounter} 인자와 $+$ 꺾쇠 괄호는 선택옵션이다. \texttt{inCounter}가 증가할 때마다 \texttt{inCounter} 를 사용해서 \texttt{counterName} 을 재설정한다. (예를 들어,  \texttt{subsection} 은 ``inCounter'' \texttt{section} 을 갖고 있다).
\end{frame}

\begin{frame}  \frametitle{변형하기}
기존 혹은 신규 계수기를 변형할 수 있다.
\begin{itemize}
\item[] {\color{command}$\backslash$setcounter\color{braces}$\{${\color{black}counter}$\}\{${\color{black}n}$\}$}
\item[] {\color{command}$\backslash$addtocounter\color{braces}$\{${\color{black}counter}$\}\{${\color{black}n}$\}$}
\item[] {\color{command}$\backslash$stepcounter\color{braces}$\{${\color{black}counter}$\}$}
\item[] {\color{command}$\backslash$refstepcounter\color{braces}$\{${\color{black}counter}$\}$}
\end{itemize}
만약 {\color{command}$\backslash$label} 을 붙여놓는다면, {\color{command}$\backslash$refstepcounter} 명령어를 사용해서 계수기 값을 참조한다.
\end{frame}

\begin{frame}  \frametitle{출력하기}
계수기를 생성하고, 변형하고, 참조할 수 있다. 하지만, 문서에서 계수기를 출력할 필요도 있다. 계수기를 다음 명령어 중 하나를 호출해서 출력한다:
\newcounter{temp}
\setcounter{temp}{4}
\begin{itemize}
\item[] {\color{command}$\backslash$arabic}{\color{braces}$\{${\color{black}chapter}$\}$} (\arabic{temp}, 아라비아 숫자)
\item[] {\color{command}$\backslash$Roman} (\Roman{temp}, 대문자 로마 숫자)
\item[] {\color{command}$\backslash$roman} (\roman{temp}, 소문자 로마 숫자)
\item[] {\color{command}$\backslash$Alph} (\Alph{temp}, 영문 대문자)
\item[] {\color{command}$\backslash$alph} (\alph{temp}, 영문 소문자)
\item[] {\color{command}$\backslash$fnsymbol} (\fnsymbol{temp}, 각주 기호)
\end{itemize}
사용자 정의 명령어와 환경에 계수기를 두고 사용한다.
\end{frame}

\subsection[명령어]{명령어}

\begin{frame}  \frametitle{간단한 명령어}
\newcommand{\xvec}{x_1, \ldots, x_n}
$x_1, ..., x_n$ 같은 흔한 문장은 다음 new 명령어로 축약할 수 있다.
\begin{itemize}
\item[] {\color{command}$\backslash$newcommand\color{braces}$\{${\color{command}$\backslash$xvec}$\}\{${\color{black}x\_1,{\color{command}$\backslash$dots},x\_n}$\}$}
\end{itemize}
상기 명령어를 삽입하고 나서 (나중에 문서에) {\color{braces}\$\color{command}$\backslash$xvec\color{braces}\$} 을 타이핑하면, $\xvec$ 을 얻게 된다. 만약 달러 부호를 생략하면 문제가 된다. 이것을 해결하는데 추가 명령어를 사용한다.
\begin{itemize}
\item[] {\color{command}$\backslash$newcommand\color{braces}$\{${\color{command}$\backslash$xvec}$\}\{${\color{command}$\backslash$ensuremath{\color{braces}$\{$}{\color{black}x\_1,{\color{command}$\backslash$dots},x\_n}}$\}$ $\}$}
\end{itemize}
두번째 정의 말미에 추가 공백을 두어서, 공백 문제를 방지하게 했다. 좀더 멋진 해결책은 {\color{highlight}xspace} 팩키지를 사용하는 것이다 (\textit{Guide to LaTeX}, 186쪽을 참조한다).
\end{frame}

\newcommand{\subvec}[2]{\ensuremath{#1_{1}, \ldots, #1_{#2}} }
\begin{frame}  \frametitle{인자를 갖는 명령어}
만약 작성한 명령어를 일반화하려고 하면, 인자 두개를 추가한다.
\begin{itemize}
\item[] \hspace{-1mm}{\color{command}$\backslash$newcommand\color{braces}$\{${\color{command}$\backslash$subvec}$\}${\color{black}[2]}$\{${\color{command}$\backslash$ensuremath{\color{braces}$\{$}{\color{black}\#1\_1,{\color{command}$\backslash$dots},\#1\_{\color{braces}$\{$}\#2{\color{braces}$\}$}}}$\}$ $\}$}
\end{itemize}

{\color{command}$\backslash$subvec\color{braces}$\{${\color{black}y}$\}\{${\color{black}m}$\}$}로부터 \subvec{y}{m} 을 생성할 수 있다.

\vspace{7mm}

$n^{th}$ 인자에 대해서 \#n 을 통해 추가 인자를 생성하고 참조할 수 있다. 디폴트 선택옵션 인자도 활용할 수 있다 (\textit{Guide to LaTeX}, 188쪽을 참조한다).
\end{frame}

\begin{frame}  \frametitle{일반화}
신규 명령어에 대한 일반적인 프레임워크는 다음과 같다.
\vspace{0.5mm} \\
\begin{itemize}
\item[] {\color{command}$\backslash$newcommand\color{braces}$\{${\color{command}$\backslash$commandName}$\}${\color{black}[n]}$\{${\color{highlight}the commands}$\}$}
\end{itemize}
\vspace{0.5mm}
여기서
\vspace{0.5mm} \\
\begin{itemize}
\item \texttt{commandName}은 명령어 명칭이 된다.
\item \texttt{n}은 인자 갯수가 된다.
\item 인자를 {\color{highlight}the commands} 에 \#1, \#2, \dots, \#n 으로 참조한다.
\end{itemize}
\vspace{0.5mm}
이미 존재하는 명령어를 다시 정의하려면, 상기와 동일한 형식으로 {\color{command}$\backslash$renewcommand} 명령어를 사용한다.
\end{frame}

\subsection[환경]{환경}

\begin{frame}  \frametitle{견본 환경}
환경은 begin과 end 태그를 사용한다 (예를 들면, \texttt{itemize}). {\color{command}$\backslash$begin} 과 {\color{command}$\backslash$end} 태그에 발생하는 것만 정의할 필요가 있다. 예를 들어, 
\vspace{-2mm}
\begin{itemize}
\item[] {\color{command}$\backslash$newenvironment\color{braces}$\{${\color{black}예제}$\}$}
\item[] \hspace{2mm}{\color{braces}$\{${\color{command}$\backslash$small$\backslash$textbf}$\{${\color{black}예제.}$\}$ {\color{command}$\backslash$hspace}$\{${\color{black}2mm}$\}\}$ {\color{red}\% 시작하는 것}}
\item[] \hspace{2mm}{\color{braces}$\{${\color{command}$\backslash\backslash$}$\}$ {\color{red}\% 끝나는 것}}
\end{itemize}
견본 환경 호출:
\vspace{-2mm}
\begin{itemize}
\item[] {\color{command}$\backslash$begin\color{braces}$\{${\color{black}예제}$\}$}
\item[] 나머지 연산의 덧셈은 신비롭게 동작한다: {\color{braces}\$}2+2=1{\color{braces}\$} (mod 3).
\item[] {\color{command}$\backslash$end\color{braces}$\{${\color{black}예제}$\}$}
\end{itemize}
\vspace{-2mm}
결과:
\vspace{-1mm}\\
\small\textbf{예제.}\hspace{1.5mm} 나머지 연산의 덧셈은 신비롭게 동작한다: $2+2 = 1$ (mod 3).
\end{frame}

\begin{frame}  \frametitle{General environment}
Generally environments take the form
\vspace{0.5mm} \\
\begin{itemize}
\item[] {\color{command}$\backslash$newenvironment\color{braces}$\{${\color{black}environmentName}$\}\{${\color{highlight}begin stuff}$\}\{${\color{highlight}end stuff}$\}$}
\end{itemize}
\vspace{0.5mm}
We can also declare that there will be \texttt{n} arguments.
\vspace{0.5mm} \\
\begin{itemize}
\item[] {\color{command}$\backslash$newenvironment\color{braces}$\{${\color{black}environmentName}$\}${\color{black}[n]}$\{${\color{highlight}begin stuff}$\}\{${\color{highlight}end stuff}$\}$}
\end{itemize}
\vspace{0.5mm}
As before, we refer to the arguments as \#1, \dots, \#n in the {\color{highlight}begin stuff} and {\color{highlight}end stuff}.
\vspace{7mm} \\
To redefine an environment that already exists, use {\color{command}$\backslash$renewenvironment} with the same format as above.
\end{frame}

\begin{frame}  \frametitle{Environment + counter}
\begin{itemize}
\item[] {\color{command}$\backslash$newcounter\color{braces}$\{${\color{black}example}$\}$}
\item[] {\color{command}$\backslash$setcounter\color{braces}$\{${\color{black}example}$\}$$\{${\color{black}0}$\}$}
\item[] {\color{command}$\backslash$newenvironment\color{braces}$\{${\color{black}example}$\}$}
\item[] \hspace{4mm}{\color{braces}$\{${\color{command}$\backslash$refstepcounter\color{braces}$\{${\color{black}example}$\}$}\color{command}$\backslash$small}
\item[] \hspace{8mm}{\color{braces}{\color{command}$\backslash$textbf}$\{${\color{black}Example} {\color{command}$\backslash$arabic\color{braces}$\{${\color{black}example}$\}$}{\color{black}.}$\}${\color{command}$\backslash$hspace}$\{${\color{black}2mm}$\}\}$}
\item[] \hspace{4mm}{\color{braces}$\{${\color{command}$\backslash\backslash$}$\}$}
\end{itemize}
\end{frame}

